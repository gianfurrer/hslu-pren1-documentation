\newacronym
  {pren1}                % id
  {PREN 1}                % display name
  {Produktentwicklung 1}  % full acronym name
  
\newacronym
  {pren2}                % id
  {PREN 2}                % display name
  {Produktentwicklung 2}  % full acronym name

\newacronym
  {yaml}
  {YAML}
  {YAML Ain't Markup Language}

\newacronym
  {tof-sensor}
  {ToF Sensor}
  {Time-of-Flight Sensor}

\newglossaryentry{i2c}{
    name={I\textsuperscript{2}C},
    description={
        Eine serielle Kommunikationsschnittstelle, die den Datenaustausch zwischen verschiedenen Komponenten wie Mikrocontrollern, Sensoren und Aktoren über nur zwei Leitungen ermöglicht: \textit{Serial Data Line} für die Datenübertragung und \textit{Serial Clock Line} für die Synchronisation. 
        Die I\textsuperscript{2}C-Schnittstelle unterstützt mehrere Geräte in einem Netzwerk und verwendet Adressen, um einzelne Komponenten anzusprechen.
    }
}

\newglossaryentry{uart}{
    name={UART},
    description={
        Abkürzung für \textit{Universal Asynchronous Receiver Transmitter}. 
        Eine Hardware-Komponente oder ein Kommunikationsprotokoll, das zur seriellen, asynchronen Datenübertragung verwendet wird. 
        UART ermöglicht die Kommunikation zwischen zwei Geräten, indem Daten über eine Sendeleitung (\textit{TX}) und eine Empfangsleitung (\textit{RX}) übertragen werden. Es erfordert keine gemeinsame Taktleitung und verwendet stattdessen Start- und Stoppbits zur Synchronisation. 
    }
}