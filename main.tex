\documentclass[oneside]{modern}
\title{Produktentwicklung 1}
\author{
\begin{tabular}{ l l l}
  \textbf{Gruppe 5} && \\
  Leu Silas & <silas.leu@stud.hslu.ch> & Maschinentechnik \\
  Rölli Silvan & <silvan.roelli.01@stud.hslu.ch> & Maschinentechnik \\
  Dietsche Thomas & <thomas.dietsche@stud.hslu.ch> & Elektrotechnik \\
  Bisang Joel & <joel.bisang@stud.hslu.ch> & Elektrotechnik \\
  Francke Patrick & <patrick.francke@stud.hslu.ch> & Informatik \\
  Furrer Gian & <gian.furrer@stud.hslu.ch> & Informatik \\
  \\
  \textbf{Betreuender Dozent} && \\
  Thalmann Markus & <markus.thalmann@hslu.ch> & Elektrotechnik \\
  \\
  \\
  \\
  TA.BA\_PREN1.H2401 &&\\
\end{tabular}
}


\newcommand{\sectionnumbering}[1]{%
  \setcounter{section}{0}%
   \renewcommand{\thesection}{\csname #1\endcsname{section}}
   \renewcommand{\thesubsection}{\thesection.\csname #1\endcsname{subsection}}}

\usepackage{etoc}
\etocsettocstyle
    {\section *{\contentsname
%                \@mkboth {\MakeUppercase \contentsname}
%                         {\MakeUppercase \contentsname}
         }
         }
    {}   

% Custom Packages
\usepackage{pdflscape}
\usepackage{booktabs}
\usepackage{graphicx}    % Für das Einfügen von Grafiken
\usepackage{subcaption}   % Für Unterabbildungen in einer Matrix
\usepackage{caption}      % Für Bildunterschriften


\begin{document}

   % magic * chapter starts at 1 :) 
   \renewcommand{\thesection}{\arabic{section}}
   % also break urls
   \makeatletter
   \g@addto@macro{\UrlBreaks}{\UrlOrds}
   \makeatother
   
   %\textcolor{light-gray}{\rule{\linewidth}{1pt}}

   %\PassOptionsToPackage{hyphens}{url}\usepackage{hyperref}

  % Print whole bibliographie
  \nocite{*}

  \firstpage
    {Pfadfinder}
    {Hochschule Luzern - Technik und Architektur}
    {\theauthor}

  %\subfile{parts/Redlichkeiterklärung.tex}
  %\subfile{parts/management_summary.tex}
  
  \newpage
  \etocdepthtag.toc{mtchapter}
  \etocsettagdepth{mtchapter}{subsection}
  \etocsettagdepth{mtappendix}{none}
  \addtableofcontents
  


  \newpage
  
  \subfile{glossary.tex}

  % ------------------------------------------------------------------------------
  % Assemble the document with the multiple parts
  
  \subfile{parts/0-doc-info}
  \subfile{parts/1-abstract}
  \subfile{parts/2-aufgabenstellung}
  \subfile{parts/3-anforderungsliste}
  \subfile{parts/4-projektplan}
  \subfile{parts/5-technologierecherche}
  \subfile{parts/6-konzeptfindung}
  \subfile{parts/7-prototyping}
  \subfile{parts/sensortest.tex}
  \subfile{parts/unrealengine}
  \subfile{parts/graphmorph}

  \newpage
  \listoffigures
  
  \newpage
  \listoftables
  
  \newpage
  \addglossary

  % Literaturverzeichnis
  \newpage
  \addcontentsline{toc}
    {section}
    {Literatur}

  \printbibliography[
    heading=subbibliography
  ]

  % Anhang
  \addcontentsline{toc}
    {section}
    {Anhang}
    
  \subfile{parts/anhang}

\end{document}