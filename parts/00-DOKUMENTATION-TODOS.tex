% NICHT TEIL DES PDFS

In diesem Dokument befinden sich Punkte, die bei der Arbeit noch angepasst werden müssen.

Gian:
- Simulator Morphologischer Kasten in Anhang
- Simulator Umsetzung dokumentieren
- Simulator Laufzeitdiagramm

Allgemein:
- Glossareinträge und Akronyme erstellen

- Wegfindung
  - Tranlation von (physischen) Wegenetzwerk zu Graph
  - Translation von Linie/Wegpunkt zu Kanten/Knoten


--- 13.12.2024

- Technologierecherche: Unterschied von Liniensensor und IR-Sensor?




FRAGEN AN DOZENT:
- Nachhaltigkeit: Auch PREN2 beschreiben?
  - Nicht unbedingt, in PREN2 gibt es eine neue Aufgabe zur Nachhaltigkeit
- Wie werden 3D gedruckte Sachen verrechnet?
  - 


--- Besprechung vom 13.12.2024 ---

UART-Signalablauf: Abbildung erklären (zB Punkt erreicht)
Passivsätze vermeiden: "Ersichtlich in Abbildung 8 ist in der Klemmmechanismus"

Klemmechanismus: "Die Feder  sorgt dafür" -> Feder wo?
- Lange Sätze vermeiden
Klemm- \& Hebemachanismus: Alle Abbildungen zuordnen
- Was wollten wir Testen, was  war das Ergebnis, was sind die Massnahmen?

Simulator:
- Morphologischer Kasten (von SImulator)



- Reflexion zur Nachhaltigkeit: Personalpronomen entfernen

- Zu lange "speziell bei Testaufbauten wurde darauf geachtet, wenn möglich, bereits
vorhandene Teile zu verwenden, beispielsweise bei Fahrtests ein Chassis aus einem alten Stück Holz zu verwenden, anstelle eines Chassis zu drucke"

- Risikomatrix Bild: Punkte überlappen
- Projektsupport: Umbenennen

Anhang:

- Morphologischer Kasten im Hauptteil verweisen
- A4.5.1 Farbsensor: Messdaten Bild zuschneiden und untereinander anordnen


Präsentation:
- Teamkollegen decken
- Klemmmechanismus wird vo jemandem erklärt, während dem der andere das Physische modell zeigt
- Hände nicht im Hosensack
- Digital-Twin: Vielleicht Video aufzeigen



Silas:
- PLA zu Glossar hinzufügen
- Abstract
- Neues Unterkapitel in Lösungskonzept für Klemmechanismus
- Feedback Thalmann
- Nachhaltigkeit: Gewählte Materialien gutschreiben

Silvan:
- Feedback Thalmann

Thomas:
- Einleitung
- Prototypen Ergebnisse: Verlinkung auf richtige Prototypen im Anhang hervorheben

Joel:
- Alle wichtigen Anhänge verlinkungen überprüfen/hinzufügen
- Feedback Thalmann

Gian:
- Simulator

Patrick:
- Feedback Thalmann
- Jede Abbildung/Tabelle auf passende Beschreibung prüfen
- Schlussdisskusion überarbeiten
  - Gemeinsames Verständnis bei Fachbegriffen war am Anfang nicht vorhanden
