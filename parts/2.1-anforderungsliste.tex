\documentclass[../main.tex]{subfiles}
\graphicspath{{\subfix{../img/}}}
\newacronym
  {pren1}                % id
  {PREN 1}                % display name
  {Produktentwicklung 1}  % full acronym name
  
\newacronym
  {pren2}                % id
  {PREN 2}                % display name
  {Produktentwicklung 2}  % full acronym name

\newacronym
  {yaml}
  {YAML}
  {YAML Ain't Markup Language}

\newacronym
  {tof-sensor}
  {ToF-Sensor}
  {Time-of-Flight Sensor}

\newglossaryentry{h-brücke}{
    name={H-Brücke},
    description={
         Eine H-Brücke ist eine Schaltung, die es ermöglicht, einen Elektromotor in beide Richtungen zu betreiben, indem sie den Stromfluss durch den Motor umkehrt. Sie besteht aus vier Schaltern (meistens Transistoren oder MOSFETs), die in einer "H"-Form angeordnet sind. Die Schalter werden so gesteuert, dass der Motor entweder vorwärts, rückwärts oder gestoppt wird.
    }
}


\newglossaryentry{pwm}{
    name={PWM},
    description={
        PWM (Pulsweitenmodulation) ist eine Technik zur Steuerung der Leistung von elektrischen Geräten, wie Motoren oder LEDs, durch das schnelle Ein- und Ausschalten eines Signals. Dabei wird die Dauer, in der das Signal "ein" ist (die Pulsbreite), im Vergleich zur Gesamtdauer eines Zyklus (der Periode) variiert.
    }
}


\newglossaryentry{ir-fototransistor}{
    name={IR-Fototransistor},
    description={
        Ein Fototransistor ist ein Halbleiterbauteil, das Licht in elektrischen Strom umwandelt. Wenn Licht auf den Transistor trifft, verändert sich seine elektrische Leitfähigkeit, was zu einer Stromänderung führt. Ein Infrarot(IR)-Fototransistor reagiert speziell auf Infrarotlicht. 
    }
}


\newglossaryentry{i2c}{
    name={I\textsuperscript{2}C},
    description={
        Eine serielle Kommunikationsschnittstelle, die den Datenaustausch zwischen verschiedenen Komponenten wie Mikrocontrollern, Sensoren und Aktoren über nur zwei Leitungen ermöglicht: \textit{Serial Data Line} für die Datenübertragung und \textit{Serial Clock Line} für die Synchronisation. 
        Die I\textsuperscript{2}C-Schnittstelle unterstützt mehrere Geräte in einem Netzwerk und verwendet Adressen, um einzelne Komponenten anzusprechen.
    }
}


\newglossaryentry{uart}{
    name={UART},
    description={
        Abkürzung für \textit{Universal Asynchronous Receiver Transmitter}. 
        Eine Hardware-Komponente oder ein Kommunikationsprotokoll, das zur seriellen, asynchronen Datenübertragung verwendet wird. 
        UART ermöglicht die Kommunikation zwischen zwei Geräten, indem Daten über eine Sendeleitung (\textit{TX}) und eine Empfangsleitung (\textit{RX}) übertragen werden. Es erfordert keine gemeinsame Taktleitung und verwendet stattdessen Start- und Stoppbits zur Synchronisation. 
    }
}


\newglossaryentry{PLA}{
    name={PLA},
    description={
    Polymilchsäure   (PLA) ist ein biologisch   abbaubarer, thermoplastischer Kunststoff, der aus erneuerbaren Ressourcen wie Maisstärke oder Zuckerrohr hergestellt wird.
    }}

\begin{document}

\newpage
\section{Anforderungsliste}
\label{sec:anforderungsliste}

In diesem Abschnitt werden alle Anforderungen an das autonome Fahrzeug aufgelistet.  
Die Anforderungen werden in zwei Kategorien eingeteilt: Festanforderung oder Wunschanforderung.  

\begin{multicols}{2}
\begin{itemize}
  \item {\bf F} = Festanforderung
  \item {\bf W} = Wunschanforderung
\end{itemize}

\columnbreak


\end{multicols}

\begin{longtable}[]{@{}lp{4.5cm}p{7.5cm}cc}
  \textbf{Nr.}
& \textbf{Anforderung}
& \textbf{Beschreibung}
& \textbf{Kat.}

\tabularnewline
\hline
\endhead
  \hypertarget{A1}{1.} & \multicolumn{4}{l}{\textbf{Allgemeine Anforderungen}} \\ \hline
  1.1 & Projektziel & Entwicklung eines autonomen Fahrzeugs, das den Weg durch ein Wegenetzwerk findet. & F\\ \hline
  1.2 & Wegfindung & Das Fahrzeug findet den optimalen Weg durch das Wegnetzwerk. & W \\ \hline
  1.3 & Autonomie & Das Fahrzeug muss nach dem Start autonom agieren, ohne externe Eingriffe. & F  \\ \hline
  1.4 & Kernfunktionalität & Erkennen und Umfahren von Hindernissen und gesperrten Wegpunkten. & F  \\ \hline
  
  2.  & \multicolumn{4}{l}{\textbf{Technische Anforderungen}} \\ \hline
  \hypertarget{A2.1}{2.1} & Dimensionen & Das Fahrzeug muss am Anfang auf eine Startfläche von 30 x 30 x 80 cm passen (das beinhaltet Anbauteile.). Diese dürfen bei der Bewältigung der Hindernisse in Breite und Länge überschritten werden. & F \\ \hline
  2.2 & Maximalgewicht & Das Fahrzeug darf maximal 2 kg wiegen. & F  \\ \hline
  \hypertarget{A2.3}{2.3} & Hardware-Integration & Alle Sensoren, Aktoren und Steuergeräte müssen im Fahrzeug integriert sein. & F\\ \hline
  \hypertarget{A2.4}{2.4} & Zielerreichungssignal & Das Fahrzeug muss visuell oder akustisch anzeigen, wenn es das Ziel erreicht. & F \\ \hline
  \hypertarget{A2.5}{2.5} & Erreichen des Ziels & Das Fahrzeug muss den Mittelpunkt des Zielkreises abdecken. & F \\ \hline
  \hypertarget{A2.6}{2.6} & Minimale Fahrgeschwindigkeit & Das Fahrzeug soll mindestens 20 cm/s auf einer geraden Strecke ohne Hindernisse fahren. & W \\ \hline
 
  3.  & \multicolumn{4}{l}{\textbf{Erkennung des Weges und der Wegpunkte}} \\ \hline
  \hypertarget{A3.1}{3.1} & Befahren des Weges & Mindestens ein Teil des Fahrzeugs muss immer auf der Linie bleiben. & F \\ \hline
  3.2 & Linienlänge & Eine Linie ist zwischen 0.5 und 2 Meter lang. & F \\ \hline
  3.3 & Linienbreite & Eine Linie ist zwischen 15 und 25 Millimeter breit. & F \\ \hline
  3.4 & Linienfarbe & Die Linien bestehen aus hellem Klebeband. & F \\ \hline
  3.5 & Wegpunkte & Ein Wegpunkt hat einen Durchmesser von 7–12 cm. & F \\ \hline
  \hypertarget{A3.6}{3.6} & Erkennung gesperrter Wegpunkte & Gesperrte Wegpunkte müssen erkannt und umfahren werden (markiert durch Pylonen). & F \\ \hline

  4.  & \multicolumn{4}{l}{\textbf{Hindernisse}} \\ \hline
  \hypertarget{A4.1}{4.1} & Erkennung und Handling von Hindernissen & Das Fahrzeug erkennt Hindernisse. Es nimmt diese auf und stellt sie an der ursprünglichen Position zurück. (Hindernisse dürfen um 180° horizontal gedreht abgesetzt werden). (Toleranzzone 20 mm umlaufend). & F\\ \hline
  4.2 & Hindernis-Anzahl pro Linie & Maximal ein Hindernis pro Linie. & F\\ \hline
  4.3 & Hindernis-Dimensionen & 135 mm x 38 mm x 60 mm (L x B x H) +/- 15 mm. & F  \\ \hline
  4.4 & Hindernis-Gewicht & 50–300 Gramm & F \\ \hline
  4.5 & Hindernis-Orientierung & Befindet sich vor dem Aufheben orthogonal und zentriert auf der Linie (+/- 15° und 2 cm). & F \\ \hline
  4.6 & Hindernis-Position auf Linie & Minimalabstand der Hindernisse zum nächsten Punkt beträgt 20 cm. & F \\ \hline

   \hypertarget{A5}{5} & \multicolumn{4}{l}{\textbf{Steuerung \& Bedienung}} \\ \hline
  5.1 & Startmethode & Start des Fahrzeugs durch physischen Schalter. & F \\ \hline
  5.2 & Zielauswahl & Ziel wird über einen Wahlschalter (A, B oder C) vor dem Start ausgewählt. & F  \\ \hline
  5.3 & Notabschaltung & Das Fahrzeug muss über einen jederzeit zugänglichen Not-Aus-Schalter verfügen. & F  \\ \hline

  6.  & \multicolumn{4}{l}{\textbf{Softwareanforderungen}} \\ \hline
  \hypertarget{A6.1}{6.1} & Simulator & Testen des Verhaltens des Fahrzeugs auf Hindernisse, gesperrte Wegpunkte und fehlende Linien mit einem Simulator. & F  \\ \hline

  7.  & \multicolumn{4}{l}{\textbf{Nachhaltigkeit}} \\ \hline
  7.1 & Umsetzung & Das Produkt soll möglichst nachhaltig entwickelt werden. & F  \\ \hline

   \hypertarget{A8}{8}  & \multicolumn{4}{l}{\textbf{Kostenrahmen}} \\ \hline
   \hypertarget{A8.1}{8.1} & Budget & Maximal CHF 500,- für das gesamte Projekt, davon maximal CHF 200,- in der Konzeptphase. & F \\ \hline
  
   \hypertarget{A9}{9} & \multicolumn{4}{l}{\textbf{Zeit}} \\ \hline
  9.1 & Maximale Dauer des Tests & Der Testlauf (ohne Vorbereitung) darf maximal 4 Minuten dauern. & F \\ \hline
  9.2 & Schnelligkeit des Fahrzeugs & Das Fahrzeug erreicht das Ziel innerhalb von 2 Minuten ohne Hindernisse. & W  \\ \hline

\caption{Anforderungsliste}
\label{tab:Anforderungsliste}
\end{longtable}

\begin{tabularx}{\textwidth}{l l X l}
        \textbf{Version} & \textbf{Datum} & \textbf{Änderung} & \textbf{Verantwortlich} \\ \hline
        1.0 & 27.09.2024 & Erstellung & Silvan Rölli \\ \hline
        1.1 & 04.10.2024 & Anpassung an FAQ & Thomas Dietsche \\ \hline
\end{tabularx}


\end{document}