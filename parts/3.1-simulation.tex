\documentclass[../main.tex]{subfiles}
\graphicspath{{\subfix{../img/}}}
\newacronym
  {pren1}                % id
  {PREN 1}                % display name
  {Produktentwicklung 1}  % full acronym name
  
\newacronym
  {pren2}                % id
  {PREN 2}                % display name
  {Produktentwicklung 2}  % full acronym name

\newacronym
  {yaml}
  {YAML}
  {YAML Ain't Markup Language}

\newacronym
  {tof-sensor}
  {ToF-Sensor}
  {Time-of-Flight Sensor}

\newglossaryentry{h-brücke}{
    name={H-Brücke},
    description={
         Eine H-Brücke ist eine Schaltung, die es ermöglicht, einen Elektromotor in beide Richtungen zu betreiben, indem sie den Stromfluss durch den Motor umkehrt. Sie besteht aus vier Schaltern (meistens Transistoren oder MOSFETs), die in einer "H"-Form angeordnet sind. Die Schalter werden so gesteuert, dass der Motor entweder vorwärts, rückwärts oder gestoppt wird.
    }
}


\newglossaryentry{pwm}{
    name={PWM},
    description={
        PWM (Pulsweitenmodulation) ist eine Technik zur Steuerung der Leistung von elektrischen Geräten, wie Motoren oder LEDs, durch das schnelle Ein- und Ausschalten eines Signals. Dabei wird die Dauer, in der das Signal "ein" ist (die Pulsbreite), im Vergleich zur Gesamtdauer eines Zyklus (der Periode) variiert.
    }
}


\newglossaryentry{ir-fototransistor}{
    name={IR-Fototransistor},
    description={
        Ein Fototransistor ist ein Halbleiterbauteil, das Licht in elektrischen Strom umwandelt. Wenn Licht auf den Transistor trifft, verändert sich seine elektrische Leitfähigkeit, was zu einer Stromänderung führt. Ein Infrarot(IR)-Fototransistor reagiert speziell auf Infrarotlicht. 
    }
}


\newglossaryentry{i2c}{
    name={I\textsuperscript{2}C},
    description={
        Eine serielle Kommunikationsschnittstelle, die den Datenaustausch zwischen verschiedenen Komponenten wie Mikrocontrollern, Sensoren und Aktoren über nur zwei Leitungen ermöglicht: \textit{Serial Data Line} für die Datenübertragung und \textit{Serial Clock Line} für die Synchronisation. 
        Die I\textsuperscript{2}C-Schnittstelle unterstützt mehrere Geräte in einem Netzwerk und verwendet Adressen, um einzelne Komponenten anzusprechen.
    }
}


\newglossaryentry{uart}{
    name={UART},
    description={
        Abkürzung für \textit{Universal Asynchronous Receiver Transmitter}. 
        Eine Hardware-Komponente oder ein Kommunikationsprotokoll, das zur seriellen, asynchronen Datenübertragung verwendet wird. 
        UART ermöglicht die Kommunikation zwischen zwei Geräten, indem Daten über eine Sendeleitung (\textit{TX}) und eine Empfangsleitung (\textit{RX}) übertragen werden. Es erfordert keine gemeinsame Taktleitung und verwendet stattdessen Start- und Stoppbits zur Synchronisation. 
    }
}


\newglossaryentry{PLA}{
    name={PLA},
    description={
    Polymilchsäure   (PLA) ist ein biologisch   abbaubarer, thermoplastischer Kunststoff, der aus erneuerbaren Ressourcen wie Maisstärke oder Zuckerrohr hergestellt wird.
    }}

\begin{document}

\newpage
\subsection{Simulator}\label{simulator}

Der Simulator dient dazu, die Funktionalität der Software des autonomen Fahrzeugs zu testen, bevor der physische Prototyp gebaut ist.
Der Sourcecode vom Simulator ist auf Github\footnote{https://github.com/gianfurrer/hslu-pren-simulator} oder im Anhang zu finden.

\subsubsection{Spezifikation}

In diesem Abschnitt wird definiert, was der Simulator genau leisten soll. Die Anforderungen stammen aus der Aufgabenstellung sowie dem gewählten Fahrzeugkonzept.

\begin{table}[htbp!]
    \centering
    \begin{tabularx}{\textwidth}{| l | X | l |}
        \hline
        \textbf{Nr.} & \textbf{Spezifikation} & \textbf{Priorität} \\ \hline
        1. & Es wird (ohne Beachtung der Hindernisse) immer der schnellste Weg ins Ziel gefunden. & Hoch \\ \hline
        2. & Entfernte Linien werden nicht befahren. & Hoch \\ \hline
        3. & Linien mit Hindernissen werden erkannt und nur befahren, falls ein Umweg länger dauert. & Hoch \\ \hline
        4. & Neue Informationen können während der Fahrt aufgenommen und entsprechende Strategie-Anpassungen dazu vorgenommen werden. & Hoch \\ \hline
        5. & Das Ziel kann konfiguriert werden (A, B oder C) & Hoch \\ \hline
        6. & Es gibt ein User Interface, welches den befahrenen Graphen aufzeigt. & Hoch \\ \hline
        6.1 & Im User Interface werden Pylonen, Hindernisse und entfernte Linien aufgezeigt & Mittel \\ \hline
        7. & Kommandos, die an die Hardware geschickt werden sollen, werden ausgegeben. & Mittel \\ \hline
        8. & Der Simulator analysiert den Graphen anhand von Bildern. & Tief \\ \hline
    \end{tabularx}
    \caption{Spezifikationen des Simulators}
\end{table}

\subsubsection{Konzept}

Das Konzept basiert auf dem Morphologischen Kasten, welcher anhand der definierten Spezifikationen erarbeitet wurde. Details dazu sind im Anhang \ref{sec:sim_Morphologischer_Kasten} zu finden.

Im Optimalfall bildet der Simulator die gesamte Hardware des autonomen Fahrzeugs nach. Dadurch könnte die Fahrzeugsoftware direkt im Simulator getestet werden, ohne Anpassungen vornehmen zu müssen. Diese Herangehensweise ermöglicht eine nahtlose Integration zwischen Softwareentwicklung und Simulation. Allerdings ist die vollständige Simulation der Hardware mit erheblichem Aufwand verbunden und übersteigt den Rahmen des Simulators.

Der Simulator ist so konzipiert, dass er anhand einer vorgegebenen Konfiguration den optimalen Weg zum Ziel findet. Diese Konfiguration beinhaltet:
\begin{itemize}
    \item Das Ziel: A, B oder C.
    \item Die Strafe (Gewichtung) für das Befahren von Linien mit einem Hindernis.
    \item Objekte die dem Simulator freigegeben werden, sobald sie im Blinkwinkel der Kamera ersichtlich sind:
     \begin{itemize}
        \item Wegpunkte mit einem Pylon 
        \item Entfernte Linien
        \item Erkannte Hindernisse
     \end{itemize}
   \item Objekte die von Kamera nicht erkannt werden. Diese werden erst beim Zufahren auf ein entsprechendes Objekt von den Hardware Sensoren erkannt und zurückgemeldet:
    \begin{itemize}
      \item Wegpunkte mit einem Pylon 
      \item Entfernte Linien
      \item Erkannte Hindernisse
    \end{itemize}
\end{itemize}

Die Informationen, die nicht von der Kamera erkannt werden, simulieren die realistische Möglichkeit, dass nicht alle Objekte von der Objekterkennung erkannt werden.  

Damit die Simulation weiss, auf welchen Wegpunkten bzw. Linien sich Hindernisse befinden, wird jeder Wegpunkt beschriftet:

\imagewidth{img/simulation/labeled-graph.png}{Beschrifteter Graph aus der Aufgabenstellung}{9cm}

\subsubsection{Konfiguration} \label{sim:Konfiguration}

Die Konfiguration wird in einer \acrshort{yaml}-Datei abgespeichert und sieht wie folgt aus.
\begin{figure}[H]
    \centering
    \begin{minted}[fontsize=\small]{yaml}
    end: B
    weight: 3
    detectable:
      cones: [E]
      obstacles: 
        - [S, F]
        - [F, A]
      removed_lines:
        - [D, B]
        - [G, C]
    undetectable:
      removed_lines:
        - [D, C]
    \end{minted}
    \label{fig:Beispielkonfiguration}
    \caption{Beispielkonfiguration für den Simulator in yaml}
\end{figure}

In der Beispielkonfiguration wird das Ziel \textbf{B} definiert. Hindernisse auf Linien erhalten eine \textbf{dreifache Gewichtung} im Vergleich zu normalen Linien.

Die Objekte im Abschnitt \textbf{detectable} können von der Kamera erkannt werden.
Die Objekte im Abschnitt \textbf{undetectable} können \textbf{nicht} von der Kamera erkannt werden, sondern erst von den Hardware Sensoren sobald das Fahrzeug direkt davor steht.
Diese Informationen werden genutzt, um den Graphen dynamisch zu aktualisieren und die optimalen Routen basierend auf den aktuellen Bedingungen neu zu berechnen.

\subsubsection{Ablauf}

In diesem Abschnitt wird der Ablauf anhand der Beispielkonfiguration in Kapitel \ref{fig:Beispielkonfiguration} erklärt.

Zu Beginn der Simulation hat der Simulator noch keine Informationen und berechnet den schnellsten Weg von S nach B: \textit{[S, E, A, B]}. Das Fahrzeug dreht sich somit zum nächsten Knoten \textit{E}.

\imagewidth{img/simulation/SimulatorAblauf-1.png}{Simulator: Fahrzeug schaut zu Wegpunkt E}{0.7\textwidth}

Um zu überprüfen, ob sich Objekte auf der idealen Strecke befindet, schiesst die Kamera ein Bild. In dem Kamerafeld (120° Winkel) sind auf der aktuellen Position \textit{S} mit dem Winkel richtung Wegpunkt \textit{E} folgende Objekte erkennbar:
\begin{itemize}
    \item Auf dem \textbf{Knoten E} befindet sich eine Pylone.
    \item Auf der \textbf{Kante S-F} befindet sich ein Hindernis.
    \item Auf der \textbf{Kante F-A} befindet sich ein Hindernis.
\end{itemize}

Anhand von diesen Informationen wird der schnellste Weg neu berechnet: \textit{[S, G, D, B]}. Das Fahrzeug dreht sicher somit zu Wegpunkt \textit{G}. 

\imagewidth{img/simulation/SimulatorAblauf-2.png}{Simulator: Fahrzeug dreht sich zu Wegpunkt G}{0.7\textwidth}

Die Kamera schiesst erneut ein Bild und erhält neue Informationen:
\begin{itemize}
    \item Die \textbf{Kante G-C} ist nicht vorhanden.
\end{itemize}

Der schnellste Weg bleibt unverändert, auch nach Einbezug neuer Informationen.
Deshalb fährt das Fahrzeug nun von Wegpunkt \textit{S} zu \textit{G}.

\imagewidth{img/simulation/SimulatorAblauf-3.png}{Simulator: Fahrzeug fährt zu Wegpunkt G}{0.7\textwidth}

Dieser Ablauf wiederholt sich, bis das Fahrzeug im Ziel ankommt:

\begin{itemize}
    \item Das Fahrzeug dreht sich zu Wegpunkt \textit{D} und entdeckt neue Informationen:
    \begin{itemize}
        \item Die \textbf{Kante D-B} ist nicht vorhanden.
    \end{itemize}
    \item Es wird ein neuer Weg berechnet, wobei der Wegpunkt \textit{D} als nächste Option gleich bleibt: \textit{[G, D, C, B]}. Somit wird der Wegpunkt \textit{D} befahren.
    \item Das Fahrzeug dreht sich zu Wegpunkt \textit{C} und entdeckt keine neue Objekte (siehe Abbildung \ref{fig:img/simulation/SimulatorAblauf-4.png}).
    \item Das Fahrzeug versucht loszufahren, aber die Sensoren finden die Linie \textit{[D, C]} nicht.
    \item Die entfernte Linie \textit{[D, C]} wird gemeldet und der schnellste Weg neu berechnet: \textit{[D, A, B]}
    \item Das Fahrzeug dreht sich zu Wegpunkt \textit{A} und entdeckt keine neue Objekte (siehe Abbildung \ref{fig:img/simulation/SimulatorAblauf-5.png}).
    \item Das Fahrzeug fährt zu Wegpunkt \textit{A}.
    \item Das Fahrzeug dreht sich zu Wegpunkt \textit{B} und entdeckt keine neue Objekte (siehe Abbildung \ref{fig:img/simulation/SimulatorAblauf-6.png}).
    \item Das Fahrzeug fährt zu Wegpunkt \textit{B}.
    \item Das Fahrzeug befindet sich auf dem Ziel, wodurch ein Zielsignal aufleuchetet (siehe Abbildung \ref{fig:img/simulation/SimulatorAblauf-7.png}).
\end{itemize}

\begin{minipage}{.5\textwidth}
  \centering
  \imagewidth{img/simulation/SimulatorAblauf-4.png}{Simulator: Fahrzeug fährt nach D und dreht sich zu C}{0.8\textwidth}
\end{minipage}%
\begin{minipage}{.5\textwidth}
  \centering
  \imagewidth{img/simulation/SimulatorAblauf-5.png}{Simulator: Fahrzeug dreht sich zu A}{0.8\textwidth}
\end{minipage}
\begin{minipage}{.5\textwidth}
  \centering
  \imagewidth{img/simulation/SimulatorAblauf-6.png}{Simulator: Fahrzeug fährt nach A und dreht sich zu B}{0.8\textwidth}
\end{minipage}%
\begin{minipage}{.5\textwidth}
  \centering
  \imagewidth{img/simulation/SimulatorAblauf-7.png}{Simulator: Fahrzeug fährt nach B und gibt das Zielsignal}{0.8\textwidth}
\end{minipage}

\paragraph{Ablaufdiagramm}
In der Abbildung \ref{fig:img/simulation/Simulator_Laufzeitdiagramm.png} ist ein Laufzeitdiagramm des Simulators abgebildet. Darauf werden die wichtigsten Abläufe des Simulators aufgezeigt. 

\imagewidth{img/simulation/Simulator_Laufzeitdiagramm.png}{Laufzeitdiagramm des Simulators}{\textwidth}

\subsubsection{Verwendete Tools und Bibliotheken}

Der Simulator wurde mit Python3 umgesetzt.

In der Tabelle \ref{tab:simulator_bibliotheken} werden die verwendeten Python3 Bibliotheken aufgelistet:

\begin{table}[htbp!]
    \centering
    \begin{tabularx}{\textwidth}{| l | X |}
        \hline
        \textbf{Bibliothek} & \textbf{Verwendungszweck} \\ \hline
        Poetry & Zum Verwalten von Abhängigkeiten und Erstellen einer Python-Umgebung. \\ \hline
        PyYAML & Zum Einlesen der YAML-Konfigurationsdateien. \\ \hline
        NetworkX & Zur Erstellung vom Graphen und Berechnung des kürzesten Weges. \\ \hline
        Matplotlib & Für das Graphische Benutzer-Interface. \\ \hline
    \end{tabularx}
    \label{tab:simulator_bibliotheken}
    \caption{Verwendete Bibliotheken im Simulator}
\end{table}


\end{document}