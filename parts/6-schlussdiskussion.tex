\documentclass[../main.tex]{subfiles}
\graphicspath{{\subfix{../img/}}}

\begin{document}

\newpage
\section{Schlussdiskussion}

Abschliessend werden die Anforderungen an unser Konzept, die angefallenen und offenen Kosten und die Risiken aufgeführt. 
\subsection{Anforderungen}
Die Anforderungen werden wie im Kapitel 3 in die Bereiche technische Anforderungen, Weg- und Wegpunkterkennung und Hindernisbewältigung aufgeteilt. Hier wird aufgeführt, welche Anforderungen erfüllt werden und welche nicht und weshalb.
\subsubsection{Technische Anforderungen}
\subsubsection{Weg- und Wegpunkterkennung}
\subsubsection{Hindernisbewältigung}
\subsection{Risiken}
Die grössten Risiken sind Lieferengpässe, zu wenig Leistung für die Bilderkennung, Wegerkennung und die Bildqualität. 

Um nicht durch Lieferengpässen Zeit zu verlieren, werden alle benötigten Baugruppen bereits vor dem Start von PREN2 evaluiert. Weiter sollen alle mechanischen Baugruppen und elektro Layouts (PCB) bereits zu Beginn von PREN2 vorhanden sein. Dadurch kann direkt in der ersten Schulwoche das ganze Material bestellt werden.

Durch das Verwenden vom Raspberry Pi 5 und dem TinyK22, kann, falls das Raspberry Pi 5 zu wenig Leistung zur Verfügung hat, Arbeit an das TinyK22 übergeben werden. Zum Beispiel die Verarbeitung der Sensordaten. 

Das Fahrzeug soll zwischen zwei Wegpunkten gerade fahren. Dadurch ist die Linienerkennung nicht die Hauptfunktion, um auf der Linie zu bleiben. Die Linienerkennung wird als Überprüfung verwendet und um einen Wegpunkt zu erkennen. Das Risiko einen Wegpunkt mit einer Bodenfuge zu verwechseln ist durch den Grössenunterschied sehr gering. 

Die Bildqualität ist durch die starke Bewegung während der Fahrt gefährdet. Durch das Trainieren der Bildverarbeitung bei Bewegung kann das Risiko minimiert werden. Ebenfalls soll möglichst nur im Stillstand die Bilderkennung stattfinden. 

\subsection{Kosten}
Die Hardware für die Bildverarbeitung wurde in PREN1 beschaffen. Dazu zählt ein Raspberry Pi 5 8GB eine Raspberry Pi 3 Kamera und das Kabel, um die Kamera mit dem Raspberry Pi zu verbinden. Weiter wurden Motoren, eine H-Brücke, IR-LED und Fototransistoren erworben. Das Raspberry Pi 5 und die Kamera wurden für Test der Bildverarbeitung gekauft. Die Motoren werden ebenfalls in PREN1 getestet. Die IR-LED und die Fototransistoren werden für den Liniensensor benötigt. Das PCB für den Liniensensor wird ende PREN1 bestellt. Die Gesamtkosten betragen  \textcolor{red}{\textbf{XX}} CHF. Die Übersicht über die in PREN1 getätigten Ausgaben ist in der Tabelle \textcolor{red}{\textbf{XX}} ersichtlich. 

In PREN2 muss die Sensorik bestellt werden. Solche Elektronik kann sehr Kostengünstig erworben werden. Aus Erfahrung schätzen wir die Ausgaben die elektrischen Bauteile und Baugruppen auf 50 CHF. Weiter muss das PCB des Adapterbaords bestellt werden. Das Adapterboad wird in PREN2 entwickelt. Auf dem Adapterboard werden die benötigten Spannungen generiert und es verbindet die benötigte Hardware miteinander. 

Das komplette Gehäuse, die werwendeten Räder und der Hindernisbewältigungsmechanismus wird in PREN2 hergestellt. Anfallenden Kosten werden auf \textcolor{red}{\textbf{XX}} CHF geschätzt.

Folglich wird \textcolor{red}{\textbf{XX}} CHF des 500 CHF Budget benötigt. Somit ist ein Polster von \textcolor{red}{\textbf{XX}} CHF vorhanden für allfällige Mehrkosten oder Falscheinschätzungen.

\subsection{Reflexion}

In der PREN Gruppe 5 herrscht Harmonie.

\end{document}