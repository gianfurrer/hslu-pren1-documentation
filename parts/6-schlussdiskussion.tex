\documentclass[../main.tex]{subfiles}
\graphicspath{{\subfix{../img/}}}

\begin{document}

\newpage
\section{Schlussdiskussion}
\subsection{Anforderungen}
\subsubsection{Technische Anforderungen}
\subsubsection{Weg- und Wegpunkterkennung}
\subsubsection{Hindernisbewältigung}
\subsection{Risiken}

\subsection{Kosten}
Die Hardware für die Bildverarbeitung wurde in PREN1 beschaffen. Dazu zählt ein Raspberry Pi 5 8GB eine Raspberry Pi 3 Kamera und das Kabel, um die Kamera mit dem Raspberry Pi zu verbinden. Weiter wurden Motoren, eine H-Brücke, IR-LED und Fototransistoren erworben. Das Raspberry Pi 5 und die Kamera wurden für Test der Bildverarbeitung gekauft. Die Motoren werden ebenfalls in PREN1 getestet. Die IR-LED und die Fototransistoren werden für den Liniensensor benötigt. Das PCB für den Liniensensor wird ende PREN1 bestellt. Die Gesamtkosten betragen  \textcolor{red}{\textbf{XX}} CHF. Die Übersicht über die in PREN1 getätigten Ausgaben ist in der Tabelle \textcolor{red}{\textbf{XX}} ersichtlich. 

In PREN2 muss die Sensorik bestellt werden. Solche Elektronik kann sehr Kostengünstig erworben werden. Aus Erfahrung schätzen wir die Ausgaben die elektrischen Bauteile und Baugruppen auf 50 CHF. Weiter muss das PCB des Adapterbaords bestellt werden. Das Adapterboad wird in PREN2 entwickelt. Auf dem Adapterboard werden die benötigten Spannungen generiert und es verbindet die benötigte Hardware miteinander. 

Das komplette Gehäuse, die werwendeten Räder und der Hindernisbewältigungsmechanismus wird in PREN2 hergestellt. Anfallenden Kosten werden auf \textcolor{red}{\textbf{XX}} CHF geschätzt.

Folglich wird \textcolor{red}{\textbf{XX}} CHF des 500 CHF Budget benötigt. Somit ist ein Polster von \textcolor{red}{\textbf{XX}} CHF vorhanden für allfällige Mehrkosten oder Falscheinschätzungen.




\subsection{Reflexion}

\end{document}