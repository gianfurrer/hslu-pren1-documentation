\documentclass[../main.tex]{subfiles}
\graphicspath{{\subfix{../img/}}}
\newacronym
  {pren1}                % id
  {PREN 1}                % display name
  {Produktentwicklung 1}  % full acronym name
  
\newacronym
  {pren2}                % id
  {PREN 2}                % display name
  {Produktentwicklung 2}  % full acronym name

\newacronym
  {yaml}
  {YAML}
  {YAML Ain't Markup Language}

\newacronym
  {tof-sensor}
  {ToF-Sensor}
  {Time-of-Flight Sensor}

\newglossaryentry{h-brücke}{
    name={H-Brücke},
    description={
         Eine H-Brücke ist eine Schaltung, die es ermöglicht, einen Elektromotor in beide Richtungen zu betreiben, indem sie den Stromfluss durch den Motor umkehrt. Sie besteht aus vier Schaltern (meistens Transistoren oder MOSFETs), die in einer "H"-Form angeordnet sind. Die Schalter werden so gesteuert, dass der Motor entweder vorwärts, rückwärts oder gestoppt wird.
    }
}


\newglossaryentry{pwm}{
    name={PWM},
    description={
        PWM (Pulsweitenmodulation) ist eine Technik zur Steuerung der Leistung von elektrischen Geräten, wie Motoren oder LEDs, durch das schnelle Ein- und Ausschalten eines Signals. Dabei wird die Dauer, in der das Signal "ein" ist (die Pulsbreite), im Vergleich zur Gesamtdauer eines Zyklus (der Periode) variiert.
    }
}


\newglossaryentry{ir-fototransistor}{
    name={IR-Fototransistor},
    description={
        Ein Fototransistor ist ein Halbleiterbauteil, das Licht in elektrischen Strom umwandelt. Wenn Licht auf den Transistor trifft, verändert sich seine elektrische Leitfähigkeit, was zu einer Stromänderung führt. Ein Infrarot(IR)-Fototransistor reagiert speziell auf Infrarotlicht. 
    }
}


\newglossaryentry{i2c}{
    name={I\textsuperscript{2}C},
    description={
        Eine serielle Kommunikationsschnittstelle, die den Datenaustausch zwischen verschiedenen Komponenten wie Mikrocontrollern, Sensoren und Aktoren über nur zwei Leitungen ermöglicht: \textit{Serial Data Line} für die Datenübertragung und \textit{Serial Clock Line} für die Synchronisation. 
        Die I\textsuperscript{2}C-Schnittstelle unterstützt mehrere Geräte in einem Netzwerk und verwendet Adressen, um einzelne Komponenten anzusprechen.
    }
}


\newglossaryentry{uart}{
    name={UART},
    description={
        Abkürzung für \textit{Universal Asynchronous Receiver Transmitter}. 
        Eine Hardware-Komponente oder ein Kommunikationsprotokoll, das zur seriellen, asynchronen Datenübertragung verwendet wird. 
        UART ermöglicht die Kommunikation zwischen zwei Geräten, indem Daten über eine Sendeleitung (\textit{TX}) und eine Empfangsleitung (\textit{RX}) übertragen werden. Es erfordert keine gemeinsame Taktleitung und verwendet stattdessen Start- und Stoppbits zur Synchronisation. 
    }
}


\newglossaryentry{PLA}{
    name={PLA},
    description={
    Polymilchsäure   (PLA) ist ein biologisch   abbaubarer, thermoplastischer Kunststoff, der aus erneuerbaren Ressourcen wie Maisstärke oder Zuckerrohr hergestellt wird.
    }}

\begin{document}

\newpage
\section{Schlussdiskussion}

Abschliessend werden die Anforderungen an unser Konzept, die angefallenen und offenen Kosten und die Risiken aufgeführt. 
\subsection{Anforderungen}
Die Anforderungen werden wie im Kapitel \ref{sec:anforderungsliste} in die Bereiche technische Anforderungen, Weg- und Wegpunkterkennung und Hindernisbewältigung aufgeteilt. Hier wird aufgeführt, welche Anforderungen erfüllt werden und welche nicht und weshalb.

\subsubsection{Technische Anforderungen}
Zum jetzigen Zeitpunkt ist es möglich zusagen, dass die Anforderungen \hyperlink{A2.1}{2.1},\hyperlink{A2.3}{2.3}, \hyperlink{A2.4}{2.4} (siehe Kapitel \ref{sec:anforderungsliste}) erfüllt werden. Wie viel das komplette Fahrzeug schlussendlich wiegen wird, ist schwer abzuschätzen. Das Gewicht der verwendeten Bauteile hat für den Entscheid Einfluss. Da das Design und das Material des Gehäuses noch nicht feststehen, kann das Endgewicht derzeit nicht genau bestimmt werden. Die ausgewählten Motoren haben genügend Newtonmeter, um die in der Anforderung \hyperlink{A2.6}{2.6} festgelegte minimale Geschwindigkeit zu erreichen. Ob jedoch mit dieser Geschwindigkeit unter den gegebenen Umständen gefahren werden kann, kann nicht abschliessend bestätigt werden.
Es wird sich am Tag des Wettbewerbs zeigen, ob die Anforderung \hyperlink{A2.5}{2.5} erfüllt wird.

\subsubsection{Weg- und Wegpunkterkennung}
Die Anforderung \hyperlink{A3.1}{3.1} wird mit dem Liniensensor erfüllt werden. Die Anforderung \hyperlink{A3.6}{3.6} mit der Raspberry Pi Kamera. Die weiteren Anforderungen für die Weg- und Wegpunkterkennung sind aus der Aufgabenstellung gegeben.

\subsubsection{Hindernisbewältigung}
Durch den verwendeten Greifmechanismus und die Sensorik werden die Anforderungen für die Hindernisbewältigung erfüllt.

\newpage
\subsection{Risiken}
Die grössten Risiken sind Lieferengpässe, zu wenig Leistung für die Bilderkennung, Wegerkennung und die Bildqualität. 

Um nicht durch Lieferengpässen Zeit zu verlieren, werden alle benötigten Baugruppen bereits vor dem Start von PREN2 evaluiert. Weiter sollen alle mechanischen Baugruppen und Elektronik Layouts (PCB) bereits zu Beginn von PREN2 vorhanden sein. Dadurch kann direkt in der ersten Schulwoche das ganze Material bestellt werden.

Durch das Verwenden vom Raspberry Pi 5 und dem TinyK22, kann, falls das Raspberry Pi 5 zu wenig Leistung zur Verfügung hat, Arbeit an das TinyK22 übergeben werden. Zum Beispiel die Verarbeitung der Sensordaten. 

Das Fahrzeug soll zwischen zwei Wegpunkten gerade fahren. Dadurch ist die Linienerkennung nicht die Hauptfunktion, um auf der Linie zu bleiben. Die Linienerkennung wird als Überprüfung verwendet und um einen Wegpunkt zu erkennen. Das Risiko einen Wegpunkt mit einer Bodenfuge zu verwechseln ist durch den Grössenunterschied sehr gering. 

Die Bildqualität ist durch die starke Bewegung während der Fahrt gefährdet. Durch das Trainieren der Bildverarbeitung bei Bewegung kann das Risiko minimiert werden. Ebenfalls soll möglichst nur im Stillstand die Bilderkennung stattfinden. 

\subsection{Abschätzung Kosten}
Die Hardware für die Bildverarbeitung wurde in PREN1 beschaffen. Dazu zählt ein Raspberry Pi 5 8GB eine Raspberry Pi 3 Kamera und das Kabel, um die Kamera mit dem Raspberry Pi zu verbinden. Weiter wurden Motoren, eine H-Brücke, IR-LED und Fototransistoren erworben. Das Raspberry Pi 5 und die Kamera wurden für Test der Bildverarbeitung gekauft. Die Motoren werden ebenfalls in PREN1 getestet. Die IR-LED und die Fototransistoren werden für den Liniensensor benötigt. Die Kosten von PREN1 betragen 163.66 CHF. Die Übersicht über die in PREN1 getätigten Ausgaben ist in der Kostenkontrolle im Projektmanagement in Tabelle \ref{tab:ausgaben_pren1} ersichtlich. 

In PREN2 muss die Sensorik bestellt werden. Solche Elektronik kann sehr kostengünstig erworben werden. Aus Erfahrung schätzen wir die Ausgaben für die elektrischen Bauteile und Baugruppen auf ungefähr 100 CHF. Weiter muss das PCB des Adapterbaords bestellt werden. Das Adapterboard wird in PREN2 entwickelt. Auf dem Adapterboard werden die benötigten Spannungen generiert und es verbindet die benötigte Hardware miteinander. Die Herstellungskosten werden auf 80 CHF geschätzt.

Das komplette Gehäuse, die verwendeten Räder und der Hindernisbewältigungsmechanismus wird in PREN2 hergestellt. Anfallenden Kosten werden auf 100 CHF geschätzt.

Folglich wird 444 CHF des 500 CHF Budget benötigt. Somit ist ein Polster von 56 CHF vorhanden für allfällige Mehrkosten oder Falscheinschätzungen. Eine detaillierte Auflistung der gesamten PREN Kosten ist im Projektmanagement in Tabelle \ref{tab:ausgaben_pren2} ersichtlich

\begin{table}[h!]
    \centering
    \begin{tabular}{|p{5cm}|p{3cm}|}
        \hline
        \textbf{Beschreibung} & \textbf{Kosten (CHF)} \\
        \hline
        PREN1 & 164 \\
        \hline
        PREN2 & 280 \\
        \hline
        \textbf{Gesamtkosten PREN} & \textbf{444} \\
        \hline
        \textbf{Budget} & \textbf{500} \\
        \hline
        \textbf{Verfügbarer Puffer} & \textbf{56} \\
        \hline
    \end{tabular}
    \caption{Abschätzung der Kosten für PREN}
    \label{tab:kostenuebersicht}
\end{table}

\subsection{Reflexion}

In der PREN Gruppe 5 herrscht Harmonie.
\newline
\newline
Die Zusammenarbeit in der Gruppe funktioniert sehr gut. Jedes Mitglied hat ähnliche Vorstellungen, was im Modul PREN erreicht werden soll. Dadurch gab es keine Probleme bei der Zieldefinition. Es soll ein möglichst einfaches Konzept entwickelt werden, um den Auftrag zu erfüllen. Dabei soll das Erreichen des Zielpunktes im Parcours über der dafür benötigten Zeit stehen. Es werden alle anfallenden Problem in der Gruppe besprochen. Nicht bloss freitags, bei dem wöchentlichen Treffen, sondern auch unter der Woche in unserem Gruppenchat. Einer der wichtigsten Punkte, in dem die Gruppendynamik sich verbessern kann, ist die Effizienz in Diskussionen. In der Gruppe kam es wiederholt zu langen Diskussionen über ein Thema, die jedoch nur wenig Ergebnisse lieferten. Zudem arbeiteten einzelne Mitglieder gelegentlich parallel an anderen Problemen. Dadurch konnten sie nicht an den Diskussionen teilnehmen, obwohl ihre Expertise für die Lösungsfindung wichtig gewesen wäre.

Die Konzeptentwicklung hat aufgezeigt, dass nicht alles so funktioniert wie geplant und häufig ein anderer Lösungsansatz vorhanden ist. 

Schnell waren Ideen im Raum, die Schwierigkeit ist es, den Fokus nicht zu stark auf einen Lösungsansatz zu legen. Hat man einen Tunnelblick, bleiben andere, möglicherweise bessere, Lösungsansätze unentdeckt. 




\end{document}