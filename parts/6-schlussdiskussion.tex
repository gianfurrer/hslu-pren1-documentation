\documentclass[../main.tex]{subfiles}
\graphicspath{{\subfix{../img/}}}

\begin{document}

\newpage
\section{Schlussdiskussion}

Abschliessend werden die Anforderungen an unser Konzept, die angefallenen und offenen Kosten und die Risiken aufgeführt. 
\subsection{Anforderungen}
Die Anforderungen werden wie im Kapitel in \ref{sec:anforderungsliste} in die Bereiche technische Anforderungen, Weg- und Wegpunkterkennung und Hindernisbewältigung aufgeteilt. Hier wird aufgeführt, welche Anforderungen erfüllt werden und welche nicht und weshalb.
\subsubsection{Technische Anforderungen}
Zum jetzigen Zeitpunkt ist es möglich zusagen, dass die Anforderungen 2.1, 2.3, 2.4 erfüllt werden. Wie viel das komplette Fahrzeug schlussendlich wiegen wird, ist schwer abzuschätzen. Das Gewicht der verwendeten Bauteile hat für den Entscheid Einfluss. Da jedoch noch nicht klar ist, wie genau das Gehäuse aussehen wird und aus welchem Material es bestehen wird, ist eine genaue Angabe zum Endgewicht nicht möglich. Die ausgewählten Motoren haben genügend Newtonmeter, um die in der Anforderung 2.6 festgelegte minimale Geschwindigkeit zu erreichen. Ob jedoch mit dieser Geschwindigkeit unter den gegebenen Umständen gefahren werden kann, kann nicht abschliessend bestätigt werden.
Es wird sich am Tag des Wettbewerbs zeigen, ob die Anforderung 2.5 erfüllt wird.
\subsubsection{Weg- und Wegpunkterkennung}
Die Anforderung 3.1 wird mit dem Liniensensor erfüllt werden. Die Anforderung 3.6 mit der Raspberry Pi Kamera. Die weiteren Anforderungen für die Weg- und Wegpunkterkennung sind aus der Aufgabenstellung gegeben. 
\subsubsection{Hindernisbewältigung}
Durch den verwendeten Greifmechanismus und die Sensorik werden die Anforderungen für die Hindernisbewältigung erfüllt.
\newpage
\subsection{Risiken}
Die grössten Risiken sind Lieferengpässe, zu wenig Leistung für die Bilderkennung, Wegerkennung und die Bildqualität. 

Um nicht durch Lieferengpässen Zeit zu verlieren, werden alle benötigten Baugruppen bereits vor dem Start von PREN2 evaluiert. Weiter sollen alle mechanischen Baugruppen und Elektronik Layouts (PCB) bereits zu Beginn von PREN2 vorhanden sein. Dadurch kann direkt in der ersten Schulwoche das ganze Material bestellt werden.

Durch das Verwenden vom Raspberry Pi 5 und dem TinyK22, kann, falls das Raspberry Pi 5 zu wenig Leistung zur Verfügung hat, Arbeit an das TinyK22 übergeben werden. Zum Beispiel die Verarbeitung der Sensordaten. 

Das Fahrzeug soll zwischen zwei Wegpunkten gerade fahren. Dadurch ist die Linienerkennung nicht die Hauptfunktion, um auf der Linie zu bleiben. Die Linienerkennung wird als Überprüfung verwendet und um einen Wegpunkt zu erkennen. Das Risiko einen Wegpunkt mit einer Bodenfuge zu verwechseln ist durch den Grössenunterschied sehr gering. 

Die Bildqualität ist durch die starke Bewegung während der Fahrt gefährdet. Durch das Trainieren der Bildverarbeitung bei Bewegung kann das Risiko minimiert werden. Ebenfalls soll möglichst nur im Stillstand die Bilderkennung stattfinden. 

\subsection{Kosten}
Die Hardware für die Bildverarbeitung wurde in PREN1 beschaffen. Dazu zählt ein Raspberry Pi 5 8GB eine Raspberry Pi 3 Kamera und das Kabel, um die Kamera mit dem Raspberry Pi zu verbinden. Weiter wurden Motoren, eine H-Brücke, IR-LED und Fototransistoren erworben. Das Raspberry Pi 5 und die Kamera wurden für Test der Bildverarbeitung gekauft. Die Motoren werden ebenfalls in PREN1 getestet. Die IR-LED und die Fototransistoren werden für den Liniensensor benötigt. Das PCB für den Liniensensor wird ende PREN1 bestellt. Die Gesamtkosten betragen  \textcolor{red}{\textbf{Todo}} CHF. Die Übersicht über die in PREN1 getätigten Ausgaben ist in der Tabelle \textcolor{red}{\textbf{Todo}} ersichtlich. 

In PREN2 muss die Sensorik bestellt werden. Solche Elektronik kann sehr Kostengünstig erworben werden. Aus Erfahrung schätzen wir die Ausgaben die elektrischen Bauteile und Baugruppen auf 50 CHF. Weiter muss das PCB des Adapterbaords bestellt werden. Das Adapterboad wird in PREN2 entwickelt. Auf dem Adapterboard werden die benötigten Spannungen generiert und es verbindet die benötigte Hardware miteinander. 

Das komplette Gehäuse, die werwendeten Räder und der Hindernisbewältigungsmechanismus wird in PREN2 hergestellt. Anfallenden Kosten werden auf \textcolor{red}{\textbf{Todo}} CHF geschätzt.

Folglich wird \textcolor{red}{\textbf{Todo}} CHF des 500 CHF Budget benötigt. Somit ist ein Polster von \textcolor{red}{\textbf{Todo}} CHF vorhanden für allfällige Mehrkosten oder Falscheinschätzungen.

\subsection{Reflexion}

In der PREN Gruppe 5 herrscht Harmonie.
\newline
\newline
Die Zusammenarbeit in der Gruppe funktioniert sehr gut. Jedes Mitglied hat ähnliche Vorstellungen, was im Modul PREN erreicht werden soll. Dadurch gab es keine Probleme bei der Zieldefinition. Es soll ein möglichst einfaches Konzept entwickelt werden, um den Auftrag zu erfüllen. Dabei soll das Erreichen des Zielpunktes im Parcours über der dafür benötigten Zeit stehen. Es werden alle anfallenden Problem in der Gruppe besprochen. Nicht bloss freitags, bei dem wöchentlichen Treffen, sondern auch unter der Woche in unserem Gruppenchat. Einer der wichtigsten Punkte, in dem die Gruppendynamik sich verbessern kann, ist die Effizienz in Diskussionen. Es ist mehrmals vorgekommen, dass in der Gruppe lange über ein Thema diskutiert wird, ohne grossen Output oder vereinzelt Mitglieder an einem anderen Problem arbeiten und dadurch nicht an der Diskussion teilnehmen, obwohl ihre Expertise gefragt ist.

Die Konzeptentwicklung hat aufgezeigt, dass nicht alles so funktioniert wie geplant und häufig ein anderer Lösungsansatz vorhanden ist. 

Schnell waren Ideen im Raum, die Schwierigkeit ist es, den Fokus nicht zu stark auf einen Lösungsansatz zu legen. Hat man einen Tunnelblick, bleiben andere, möglicherweise bessere, Lösungsansätze unentdeckt. 



\end{document}