\documentclass[../main.tex]{subfiles}
\graphicspath{{\subfix{../../img/}}}
\begin{document}

\newpage
\section{Technologierecherche}\label{a2:technologierecherche}

Dieses Kapitel bietet einen Überblick über die durchgeführte Technologierecherche.
Dabei wird die Gesamtfunktionalität des Fahrzeugs in folgende Teilfunktionen unterteilt:
\begin{itemize}
    \item Wegfindung 
    \item Objekterkennung (Objekterken.)
    \item Fortbewegung
    \item Hindernisbewältigung
    \item Antrieb und Orientierung (A \& O)
\end{itemize}


\subsection{Überblick}
Die folgende Tabelle listet verschiedene Themen zu jeder Teilfunktion auf, die im Rahmen der Recherche untersucht wurden. Jeder Eintrag enthält eine Quelle und eine Bewertung (0–10). Die Bewertung (BW) spiegelt die Nutzbarkeit und Relevanz der jeweiligen Quelle für die zugehörige Teilfunktion wider.

\scriptsize
\begin{longtable}{p{2cm}p{2cm}p{4cm}clcl}
%\toprule
\textbf{Teilfunktion} & \textbf{Thema} &
\textbf{Beschreibung} & \textbf{BW} & \textbf{Quelle} & \textbf{Abfragedatum} &
\textbf{Wer}\tabularnewline
%\midrule
\endhead

% Wegfindung--------------------------------------------------------------------------------------------------
Wegfindung & Algorithmus & Informationen \u00fcber Wegfindung & 8 &
\href{https://de.wikipedia.org/wiki/Pathfinding}{Link} & 02.10.2024 & Gian
\tabularnewline
Wegfindung & Algorithmus & Comparing Dijkstra’s and A* Search Algorithm & 7 &
\href{https://medium.com/@miguell.m/dijkstras-and-a-search-algorithm-2e67029d7749}{Link}
& 02.10.2024 & Gian
\tabularnewline
Wegfindung & Algorithmus & Rapidly exploring random tree & 4 &
\href{https://en.wikipedia.org/wiki/Rapidly_exploring_random_tree}{Link}
& 03.10.2024 & Gian
\tabularnewline
Wegfindung & Algorithmus & D\* & 9 &
\href{https://en.wikipedia.org/wiki/D*}{Link} & 03.10.2024 & Gian
\tabularnewline
Wegfindung & Algorithmus & Goal directed shortest path queries using precomputed cluster distances & 8 &
\href{https://publikationen.bibliothek.kit.edu/1000009512}{Link} & 04.10.2024 & Patrick
\tabularnewline
Wegfindung & Algorithmus & Floyd Warshall shortest path algorithm & 8 &
\href{https://en.wikipedia.org/wiki/Floyd%E2%80%93Warshall_algorithm}{Link} & 04.10.2024 & Patrick
\tabularnewline
Wegfindung & Algorithmus & Depth First Search (DFS) & 7 &
\href{https://en.wikipedia.org/wiki/Depth-first_search}{Link} & 04.10.2024 & Patrick
\tabularnewline
Wegfindung & Algorithmus & A decremental approach with the \* algorithm & 4 &
\href{https://doi.org/10.1016/j.measurement.2014.10.014}{Link} & 04.10.2024 & Patrick
\tabularnewline
Wegfindung & Kommunikation & Virtual Serial Port for Linux & 8 &
\href{https://stackoverflow.com/questions/52187/virtual-serial-port-for-linux}{Link} & 24.10.2024 & Gian
\tabularnewline
Wegfindung & Simulation & MicroMouseSimulator & 6 &
\href{https://github.com/mackorone/mms}{Link} & 24.10.2024 & Gian
\tabularnewline
Wegfindung & Visualisierung & Matplotlib & 7 &
\href{https://matplotlib.org/}{Link} & 24.10.2024 & Gian
\tabularnewline
Wegfindung & Graphanalyse & NetworkX & 10 &
\href{https://wiki.python.org/moin/PythonGraphLibraries}{Link} & 29.11.2024
\tabularnewline

% Objekterkennung---------------------------------------------------------------------------------------------
Objekterken. & Deep Learning & Deep Learning Frameworks & 7 &  \href{https://www.simplilearn.com/tutorials/deep-learning-tutorial/deep-learning-frameworks} {Link}&  27.09.2024 & Gian
\tabularnewline
Objekterken. & Deep Learning & TensorFlow in 100 seconds & 10 &
\href{https://www.youtube.com/watch?v=i8NETqtGHms}{Link} & 29.09.2024 & Gian
\tabularnewline
Objekterken. & Deep Learning & What is Object Detection? & 9 &
\href{https://www.ibm.com/topics/object-detection#:~:text=Object%20detection%20is%20a%20technique,imaging%20to%20self%2Ddriving%20cars.}{Link} & 02.10.2024 & Gian
\tabularnewline
Objekterken. & Deep Learning & What are convolutional neural networks? & 8 &
\href{https://www.ibm.com/topics/convolutional-neural-networks}{Link} & 02.10.2024 & Gian
\tabularnewline
Objekterken. & Deep Learning & Haar Cascade - Explained & 6 &
\href{https://medium.com/analytics-vidhya/haar-cascades-explained-38210e57970d}{Link} & 02.10.2024 & Gian
\tabularnewline
Objekterken. & Deep Learning & Region Based CNN - Wikipedia & 6 &
\href{https://en.wikipedia.org/wiki/Region_Based_Convolutional_Neural_Networks}{Link}
& 02.10.2024 & Gian
\tabularnewline
Objekterken. & Kamera & What’s the Best Raspberry Pi Camera For Your Project? & 5 &
\href{https://randomnerdtutorials.com/best-raspberry-pi-camera-for-your-project/}{Link}
& 25.10.2024 & Gian
\tabularnewline
Objekterken. & Kamera & Best camera for Raspberry Pi in 2024 & 8 &
\href{https://www.xda-developers.com/best-camera-raspberry-pi/}{Link}
& 25.10.2024 & Gian
\tabularnewline

% Fortbewegung---------------------------------------------------------------------------------------------
Fortbewegung & Beweglichkeit & Prinzip Roomba & 8 & MC-Car & 01.10.2024 & Joel
\tabularnewline
Fortbewegung & Beweglichkeit & 4-Rad Differential steering / \newline skid steer & 8 & \href{https://en.wikipedia.org/wiki/Differential_steering}{Link} /\href{https://science.howstuffworks.com/transport/engines-equipment/skid-steer2.htm}{Link} & 03.10.2024 & Silas
\tabularnewline
Fortbewegung & Beweglichkeit & Omniwheel & 9 & \href{https://de.wikipedia.org/wiki/Allseitenrad}{Link} / \href{https://www.youtube.com/watch?v=wwQQnSWqB7A}{Link} & 03.10.2024 & Silas
\tabularnewline
Fortbewegung & Beweglichkeit & Mecanum-Rad & 9 & \href{https://de.wikipedia.org/wiki/Mecanum-Rad}{Link} / \href{https://www.youtube.com/watch?v=noqBUEgyQ8A}{Link} & 04.10.2024 & Silas
\tabularnewline
Fortbewegung & Beweglichkeit & Knicklenkung & 4 & \href{https://de.wikipedia.org/wiki/Knicklenkung}{Link} & 03.10.2024 & Silas
\tabularnewline
Fortbewegung & Beweglichkeit & Achsschenkellenkung & 5 & \href{https://de.wikipedia.org/wiki/Achsschenkel}{Link} & 03.10.2024 & Silas
\tabularnewline
Fortbewegung & Beweglichkeit & Drehschemellenkung & 5 & \href{https://www.staplerberater.de/auswahlkriterien/lenkungsarten}{Link} & 03.10.2024 & Silas
\tabularnewline
Fortbewegung & Beweglichkeit & Fahrzeug abbocken und an Stelle auf Drehkranz wenden & 6 & \href{https://www.kaiserkraft.ch/hubgeraete/hub-und-verladetische/auto-niveaugeraet-mit-drehscheibe/drehscheiben-1110-mm/p/M1142876/?articleNumber=118558&lang=de_CH&customerType=B2C&lang=&infinity=ict2~net~gaw~cmp~PM_DE-shopping24-Jarvis-0~ag~~ar~~kw~~mt~&gad_source=1&gclid=CjwKCAjwgfm3BhBeEiwAFfxrGxsQhJoEWwY3dNM_OYKFg2NOgoHXLP2OeyLmOZFTVnzHt7PvNpgCbhoCACQQAvD_BwE}{Link} & 03.10.2024 & Silas
\tabularnewline

% Hindernisbewältigung----------------------------------------------------------------------------------------
Hindernis & Aufnahme & Vakuum-Sauggreifer & 5 & \href{https://www.schmalz.com/de-ch/glossar/vakuumgreifer/}{Link} & 03.10.2024 & Silas
\tabularnewline
Hindernis & Aufnahme & Vakuum-Sauggreifer & 8 & \href{https://www.youtube.com/shorts/alxwWgzSVss}{Link}& 03.10.2024 & Silvan
\tabularnewline
Hindernis & Rotation & Kran & 9 & \href{https://www.youtube.com/watch?v=VZRFHJfUkq4&feature=youtu.be}{Link}& 03.10.2024 & Silvan
\tabularnewline
Hindernis & Rotation & Kran von oben & 8 & \href{https://www.youtube.com/watch?v=J7LGSNhFTU4}{Link}& 03.10.2024 & Silvan
\tabularnewline

% Antrieb & Orientierung--------------------------------------------------------------------------------------
A \& O & Motor & Schrittmotor & 8 & \href{https://wiki.bu.ost.ch/infoportal/_media/hardware/sysp/bauteile/schrittmotor_kurz_erklaert_d.pdf}{Link} & 27.09.2024 & Thomas
\tabularnewline
A \& O & Motor & DC Motor & 7 & \href{https://www.arrow.de/research-and-events/articles/which-dc-motor-is-best-for-your-application}{Link} & 27.09.2024 & Thomas
\tabularnewline
A \& O & Motor & Brushless & 7 & \href{https://www.renesas.com/en/support/engineer-school/brushless-dc-motor-01-overview}{Link} & 27.09.2024 & Thomas
\tabularnewline
A \& O & Hardware & Raspberry Pi & 8 & \href{https://www.raspberrypi.com/documentation/computers/raspberry-pi.html}{Link} & 29.09.2024 & Thomas
\tabularnewline
A \& O & Hardware & Arduino & 5 & \href{https://arduino.cc/en/hardware#boards-1}{Link} & 29.09.2024 & Thomas
\tabularnewline
A \& O & Hardware & Tiny & 8 & HSLU & 01.10.2024 & Joel
\tabularnewline
A \& O & Hardware & ESP32 & 7 & \href{https://www.espressif.com/en/products/devkits/esp32-devkitc}{Link} & 03.10.2024 & Thomas
\tabularnewline
A \& O & Sensorik & Linienfolger & 5 & \href{https://pglu.ch/3-mit-fahrroboter-linie-folgen/?srsltid=AfmBOor3qIbdXGD1WYtV-YadIVjE2Urm7U3QGtes_IjcCzCVViC2yody}{Link} & 04.10.2024 & Joel
\tabularnewline
A \& O & Sensorik & Linienfolger & 5 & \href{https://spacehal.github.io/docs/robotik/edgeFollower}{Link} & 04.10.2024 & Joel
\tabularnewline
A \& O & Sensorik & Infrarot & 5 & \href{https://www.elektronik-kompendium.de/sites/raspberry-pi/2802011.htm}{Link} & 27.09.2024 & Thomas
\tabularnewline
A \& O & Sensorik & Ultraschall & 5 & \href{https://elektro.turanis.de/html/prj121/index.html}{Link} & 27.09.2024 & Thomas 
\tabularnewline
A \& O & Sensorik & Encoder & 7 & \href{https://www.arrow.de/research-and-events/articles/rotary-encoders-how-to-pair-with-an-arduino-board}{Link} & 29.09.2024 & Thomas
\tabularnewline
A \& O & Sensorik & Optischer Sensor & 5 & \href{https://global.sharp/products/device/lineup/data/pdf/datasheet/gp2y0e02a_e.pdf}{Link} & 01.10.2024 & Joel
\tabularnewline
A \& O & Not-Aus & Implementierung eines Not-Aus & 6 &
\href{https://www.eaton.com/ie/en-gb/markets/machine-building/service-and-support-machine-building-moem-service-eaton/blogs/emergency-stop-circuit---blogs---eaton.html}{Link} & 27.09.2024 & Thomas
\tabularnewline
A \& O & Sicherheit & mobiler Roboter & 5 & \href{https://tuprints.ulb.tu-darmstadt.de/18674/1/10.1524_auto.51.10.435.19576.pdf}{Link} & 27.09.2024 & Thomas 
\tabularnewline
A \& O & Akkumulatoren & LiPo & 8 & \href{https://www.lion-care.com/lipo-akkus-eigenschaften-vorteile-und-mehr}{Link} & 27.09.2024 & Thomas
\tabularnewline
A \& O & Akkumulatoren & Li-Ion & 7 & \href{https://poleenergy.ch/shop_content.php?coID=32}{Link} & 27.09.2024 & Thomas
\tabularnewline
A \& O & Solarpanel & Run Arduino Offgrid & 6 & \href{https://voltaicsystems.com/solar-arduino-guide/}{Link} & 29.09.2024 & Thomas
\tabularnewline
A \& O & Akkumulatoren & Nickel-Metallhydrid-Akkus \newline (Batterie) & 7 & \href{https://www.chemie.de/lexikon/Nickel-Metallhydrid-Akkumulator.html}{Link} & 29.09.2024 & Joel
\tabularnewline
\caption{Technologierecherche Übersicht}
\label{tab:technologierecherche}
\end{longtable}
\normalsize




\newpage
\subsection{Wegfindung}
\label{a2:Wegfindung}

Dieser Abschnitt enthält detaillierte Rechercheergebnisse über die Bereiche der Wegfindung. Dabei werden Vor- und Nachteile aufgezeigt und die einzelnen Algorithmen erklärt.

Die Topologie des Weges mit Start und Zielmöglichkeiten ist bereits bekannt.
Somit kann anhand eines Wegfindungsalgorithmus der schnellste Weg berechnet werden. \\ 

Zum Verständnis der Vor- und Nachteile sind grundlegende Kenntnisse über Wegfindungsalgorithmen erforderlich.

\paragraph{Dijkstra}

Findet den kürzesten Weg von einem gegebenen Startknoten zu allen anderen Knoten in einem Graphen mit nicht-negativen Gewichten.

\begin{minipage}[t]{0.48\textwidth}
\begin{items}
  \item [Vorteile]
  \item Optimierter Greedy-Algorithmus, der immer die optimale Lösung findet.
  \item Funktioniert gut in dichten Graphen, wo die Anzahl der Kanten hoch ist.
\end{items}
\end{minipage}
\hfill
\begin{minipage}[t]{0.48\textwidth}
\begin{items}
  \item [Nachteile]
  \item Nicht zielgerichtet, kann zu Zeitverschwendung führen.
  \item Erfordert die Speicherung aller Knoten im Speicher, was zu hohem Speicherbedarf führen kann.
  \item Nicht dynamisch; Änderungen am Graphen erfordern einen vollständigen Neudurchlauf des Algorithmus.
\end{items}
\end{minipage}
\paragraph{A*-Algorithmus}

Ein heuristischer Algorithmus, der den kürzesten Pfad von einem Startknoten zu einem Zielknoten findet. Er kombiniert Dijkstra mit einer Heuristik, um effizientere Entscheidungen zu treffen \cite{dijkstra_vs_astar}.

\begin{minipage}[t]{0.48\textwidth}
\begin{items}
  \item [Vorteile]
  \item Findet optimale Lösungen und ist oft schneller als Dijkstra, besonders bei grossen Graphen.
  \item Anpassbar durch Wahl der Heuristik.
\end{items}
\end{minipage}
\hfill
\begin{minipage}[t]{0.48\textwidth}
\begin{items}
  \item [Nachteile]
  \item Die Effizienz hängt stark von der Wahl der Heuristik ab.
  \item Kann in bestimmten Fällen langsamer sein als Dijkstra, wenn die Heuristik schlecht gewählt ist.
\end{items}
\end{minipage}

\paragraph{Rapidly-exploring Random Tree (RRT)}

Ein probabilistischer Algorithmus, der vor allem in der Robotik eingesetzt wird,
um schnell einen Pfad in einem komplexen Raum zu finden \cite{rrt}.

\begin{minipage}[t]{0.48\textwidth}
\begin{items}
  \item [Vorteile]
  \item Sehr effektiv in hochdimensionalen Räumen.
  \item Flexibel, da er dynamisch auf neue Hindernisse reagieren kann.
\end{items}
\end{minipage}
\hfill
\begin{minipage}[t]{0.48\textwidth}
\begin{items}
  \item [Nachteile]
  \item Kann suboptimale Lösungen liefern, da der Pfad eher zufällig als deterministisch erzeugt wird.
  \item Benötigt oft zusätzliche Post-Processing-Schritte zur Glättung des Pfades.
\end{items}
\end{minipage}

\paragraph{Dynamic A* (D* Lite)}

Eine Erweiterung des A*-Algorithmus, die speziell für dynamische Umgebungen entwickelt wurde, in denen sich Hindernisse während der Laufzeit ändern können \cite{d_star}.

\begin{minipage}[t]{0.48\textwidth}
\begin{items}
  \item [Vorteile]
  \item Effiziente Anpassung des Pfades bei Änderungen der Umgebung.
  \item Minimiert die Berechnungen, indem nur die betroffenen Teile des Pfades neu berechnet werden.
\end{items}
\end{minipage}
\hfill
\begin{minipage}[t]{0.48\textwidth}
\begin{items}
  \item [Nachteile]
  \item Kann in sehr dynamischen Umgebungen langsamer sein, wenn viele Änderungen gleichzeitig auftreten.
  \item Komplexität der Implementierung im Vergleich zu einfacheren Algorithmen.
\end{items}
\end{minipage}


\paragraph{Goal directed shortest path queries using precomputed cluster distances}

''Goal directed shortest path queries using precomputed cluster distances'' ist eine beschleunigte Version von Dijkstra's Algorithmus, wobei für Graphen-Cluster Distanzen vorberechnet werden \cite{goal_directed_queries}.

\begin{minipage}[t]{0.48\textwidth}
\begin{items}
  \item [Vorteile]
  \item Vorgenerierte kürzeste Pfade erlauben schnelle Wechsel bei Graph-Anpassungen.
  \item Schnell bei kleinen Graphen.
\end{items}
\end{minipage}
\hfill
\begin{minipage}[t]{0.48\textwidth}
\begin{items}
  \item [Nachteile]
  \item Relativ grosser Speicheraufwand für die vorberechneten kürzesten Pfade.
  \item Komplexität der Implementierung im Vergleich zu einfacheren Algorithmen.
\end{items}
\end{minipage}

\paragraph{Floyd Warshall}
Findet den kürzesten Weg von allen Knoten zu allen anderen Knoten eines Graphen \cite{floyd_warshall}.

\begin{minipage}[t]{0.48\textwidth}
\begin{items}
  \item [Vorteile]
  \item Effizient für kleine und dichte Graphen.
\end{items}
\end{minipage}
\hfill
\begin{minipage}[t]{0.48\textwidth}
\begin{items}
  \item [Nachteile]
  \item Langsamer als Dijkstra oder A* bei grossen Graphen.
\end{items}
\end{minipage}

\subsection{Objekterkennung}
\label{a2:Objekterkennung}

Der Weg, die Hindernisse sowie die Pylonen sollen anhand von Software erkannt und kategorisiert werden. Um die Objekte zu erkennen und kategorisieren, wird Objekterkennung verwendet \cite{object_detection_ibm}.

Dieser Abschnitt fasst die Rechercheergebnisse zusammen und listet Vor- und Nachteile zu unterschiedlichen Varianten.

\paragraph{Convolutional Neural Networks (CNN)}

Ein tiefes Lernverfahren, das besonders für die Bildverarbeitung und Objekterkennung geeignet ist. CNNs nutzen Faltungsschichten, um Merkmale aus Bildern zu extrahieren \cite{cnn}.

\begin{minipage}[t]{0.48\textwidth}
\begin{items}
  \item [Vorteile]
  \item Hohe Genauigkeit bei der Objekterkennung, besonders bei komplexen Szenen.
  \item Automatische Merkmalserkennung ohne manuelles Feature-Engineering.
\end{items}
\end{minipage}
\hfill
\begin{minipage}[t]{0.48\textwidth}
\begin{items}
  \item [Nachteile]
  \item Hoher Rechenaufwand und benötigte Datenmengen für das Training.
  \item Empfindlich gegenüber Veränderungen in der Beleuchtung und Bildqualität.
\end{items}
\end{minipage}

\paragraph{YOLO (You Only Look Once)}

Ein Echtzeit-Objekterkennungsalgorithmus, der in der Lage ist, mehrere Objekte in einem Bild in einem einzigen Durchgang zu erkennen. Basierend auf CNN \cite{yolo}.

\begin{minipage}[t]{0.48\textwidth}
\begin{items}
  \item [Vorteile]
  \item Sehr schnell und effizient, ideal für Echtzeitanwendungen.
  \item Kann mehrere Klassen in einem Bild gleichzeitig erkennen.
\end{items}
\end{minipage}
\hfill
\begin{minipage}[t]{0.48\textwidth}
\begin{items}
  \item [Nachteile]
  \item Geringere Genauigkeit bei kleineren Objekten im Vergleich zu anderen Methoden.
  \item Schwierigkeiten bei der Erkennung von überlappenden Objekten.
\end{items}
\end{minipage}

\paragraph{Haar-Cascade-Klassifikatoren}

Ein Algorithmus zur schnellen Objekterkennung, der häufig in Echtzeitanwendungen verwendet wird \cite{haar-cascade}.

\begin{minipage}[t]{0.48\textwidth}
\begin{items}
  \item [Vorteile]
  \item Schnell und effizient für die Erkennung bestimmter Objekte wie Gesichter oder Pylonen.
  \item Relativ einfach zu implementieren und zu trainieren.
\end{items}
\end{minipage}
\hfill
\begin{minipage}[t]{0.48\textwidth}
\begin{items}
  \item [Nachteile]
  \item Nicht so robust gegen Variationen in Beleuchtung und Hintergrund.
  \item Begrenzte Genauigkeit und Flexibilität im Vergleich zu tiefen Lernansätzen.
\end{items}
\end{minipage}

\paragraph{Region-based CNN (R-CNN)}

Ein Ansatz, der die Genauigkeit der Objekterkennung verbessert, indem er Regionenvorschläge nutzt und diese dann mit einem CNN klassifiziert \cite{rcnn_wikipedia}.

\begin{minipage}[t]{0.48\textwidth}
    \begin{items}
      \item [Vorteile]
      \item Hohe Genauigkeit bei der Erkennung von Objekten in Bildern.
      \item Gut geeignet für die Erkennung in komplexen Szenen.
    \end{items}
\end{minipage}
\hfill
\begin{minipage}[t]{0.48\textwidth}
    \begin{items}
      \item [Nachteile]
      \item Hoher Rechenaufwand, was die Echtzeitfähigkeit einschränkt.
      \item Erfordert eine sorgfältige Abstimmung der Hyperparameter und Regionenvorschläge.
    \end{items}
\end{minipage}

\newpage
\subsection{Fortbewegung}
\label{a2:Fortbewegung}

Dieser Abschnitt enthält detaillierte Rechercheergebnisse zu den unterschiedlichen Arten der Fortbewegung und deren Vor- und Nachteilen.


\subsubsection{Mecanum-Rad} \label{recherche-mecanum-rad}

Das Mecanum-Rad ist ein Rad, das einem Fahrzeug omnidirektionale Fahrmanöver erlaubt, ohne mit einer mechanischen Lenkung ausgestattet zu sein.
Auf dem Umfang des Rades sind mehrere einzelne tonnenförmige Rollen in einem Winkel von 45° gelagert angebracht. 
Je nach Antrieb der Räder kann sich das Fahrzeug in alle Richtungen bewegen und um die eigene Achse drehen\cite{mecanum_rad}.

\begin{minipage}[t]{0.48\textwidth}
    \begin{items}
      \item [Vorteile]
      \item Freie Bewegung in alle Himmelsrichtungen.
      \item Rotation um eigene Achse an Ort und Stelle.
      \item Keine mechanischen Teile für die Lenkung nötig.
 \end{items}
\end{minipage}
\hfill
\begin{minipage}[t]{0.48\textwidth}
    \begin{items}
      \item [Nachteile]
      \item Vier einzeln ansteuerbare Motoren nötig.
      \item Hoher Programmier- und Rechenaufwand.
      \item Kleine Rollen an Rad könnten Probleme verursachen (z.B. bei Plattenfugen).
    \end{items}
\end{minipage}

\subsubsection{Omniwheel} \label{recherche-omniwheel}

Omniwheels sind ähnlich aufgebaut wie Mecanum-Räder, jedoch sind die Rollen in einem 90° Winkel angeordnet \cite{omniwheel}. Dadurch bietet es sich an, drei Räder in einem Winkel von 120° am Fahrzeug anzuordnen.

\begin{minipage}[t]{0.48\textwidth}
    \begin{items}
        \item[Vorteile]
        \item Freie Bewegung in alle Himmelsrichtungen.
      \item Rotation um eigene Achse an Ort und Stelle.
      \item Keine mechanischen Teile für die Lenkung nötig.
    \end{items}
\end{minipage}
\begin{minipage}[t]{0.48\textwidth}
    \begin{items}
      \item [Nachteile]
      \item Drei einzeln ansteuerbare Motoren nötig.
      \item Hoher Programmier- und Rechenaufwand.
       \item Kleine Rollen an Rad könnten Probleme verursachen (z.B. bei Plattenfugen).
    \end{items}
\end{minipage}

\subsubsection{4-Rad Differential-steering / skid-steering}
Dieses Lenksystem wird umgangssprachlich auch als Panzersteuerung bezeichnet. Dabei sind die Räder paarweise an den Fahrzeugseiten angebracht und verfügen über keine Lenkmechanik. Die Richtungsänderung erfolgt ausschliesslich durch unterschiedliche Drehzahlen der Räder auf den beiden Seiten des Fahrzeugs. Durch gegenläufiges Drehen der Räder kann das Fahrzeug auf der Stelle rotieren. Der Antrieb der beiden Räder auf einer Seite wird in der Regel durch einen einzelnen Motor realisiert, der die Kraft über einen Riemen- oder Kettentrieb auf die Räder überträgt \cite{skid_steering}.

\begin{minipage}[t]{0.48\textwidth}
    \begin{items}
        \item[Vorteile]
        \item Stabiler Stand des Fahrzeuges.
        \item Rotation um eigene Achse an Ort und Stelle.
        \item Keine mechanischen Teile für die Lenkung nötig.
        \item Nur zwei Motoren benötigt.
        \item Geringer Programmier- und Rechenaufwand.
    \end{items}
\end{minipage}
\begin{minipage}[t]{0.48\textwidth}
    \begin{items}
      \item [Nachteile]
      \item Gefahr von Aufschaukeln des Fahrzeuges bei Drehungen um die eigene Achse.
      \item Gefahr von Schieben der langsameren Räder bei Kurvenfahrt aufgrund von fehlendem Grip.
    \end{items}
\end{minipage}

\subsubsection{Prinzip Roomba} \label{recherche-prinzip-roomba}
Der Antrieb erfolgt über zwei fest am Fahrzeug montierte Räder, die einzeln angetrieben werden. Das dritte Rad ist ein frei bewegliches Schwenkrad. Dieses System ist vor allem von Staubsaugerrobotern bekannt.

\begin{minipage}[t]{0.48\textwidth}
    \begin{items}
        \item[Vorteile]
        \item Rotation um eigene Achse an Ort und Stelle.
        \item Keine Gefahr von Aufschaukeln bei Drehungen um die eigene Achse.
        \item Keine mechanischen Teile für die Lenkung nötig.
        \item Nur zwei Motoren erforderlich.
         \item Geringer Programmier- und Rechenaufwand.
    \end{items}
\end{minipage}
\begin{minipage}[t]{0.48\textwidth}
    \begin{items}
      \item [Nachteile]
      \item Gefahr von Verfangen des kleinen Stützrades in Bodenfugen.
      \item Kippgefahr bei Hindernisaufnahme.)
    \end{items}
\end{minipage}

\subsubsection{Knicklenkung} \label{recherche-knicklenkung}

Bei einer Knicklenkung besteht das Fahrzeug aus einem Vorder- und einem Hinterwagen, welche mit einem Gelenk in der Mitte verbunden sind. Die Lenkung erfolgt lediglich über dieses Gelenk \cite{knicklenkung}.

\begin{minipage}[t]{0.48\textwidth}
    \begin{items}
      \item [Vorteile]
      \item Robuste Bauweise.
 \end{items}
\end{minipage}
\hfill
\begin{minipage}[t]{0.48\textwidth}
    \begin{items}
      \item [Nachteile]
      \item Grosser Wendekreis.
      \item Lenkzylinder oder ähnliches nötig.
    \end{items}
\end{minipage}

\subsubsection{Achsschenkellenkung} \label{recherche-achsschenkellenkung}

Gängigstes Lenksystem bei Autos, bestehend aus zwei starren Rädern und zwei gelenkten Rädern \cite{achsschenkel}.

\begin{minipage}[t]{0.48\textwidth}
    \begin{items}
      \item [Vorteile]
      \item Stabiler Stand.
 \end{items}
\end{minipage}
\hfill
\begin{minipage}[t]{0.48\textwidth}
    \begin{items}
      \item [Nachteile]
      \item Grosser Wendekreis.
      \item Viele mechanische Teile nötig.
    \end{items}
\end{minipage}

\subsubsection{Drehschemellenkung} \label{recherche-drehschemellenkung}

Bekanntes Lenksystem von Anhängern (z.B. Leiterwagen). Die Idee hier ist jedoch ein einzelnes gelenktes Rad, welches gleichzeitig auch das einzige angetriebene Rad ist. Die beiden anderen Räder werden dabei nicht angetrieben \cite{lenkungsarten}.

\begin{minipage}[t]{0.48\textwidth}
    \begin{items}
      \item [Vorteile]
      \item Einfache Bauweise.
 \end{items}
\end{minipage}
\hfill
\begin{minipage}[t]{0.48\textwidth}
    \begin{items}
      \item [Nachteile]
      \item Nur ein angetriebenes Rad (Traktionsprobleme).
    \end{items}
\end{minipage}

\subsubsection{Abbocken und drehen} \label{recherche-abbocken-und-drehen}

Die Idee ist eine nach unten absenkbare Fussplatte, welche das Fahrzeug im Zentrum abhebt und darauf um die eigene Achse dreht. Das System kann auch ergänzend zu einer anderen Lenkart verwendet werden.

\begin{minipage}[t]{0.48\textwidth}
    \begin{items}
      \item [Vorteile]
      \item Kein Verrutschen bei der Drehung.
      \item Hindernis kann mit Fahrzeug zusammen angehoben werden.
 \end{items}
\end{minipage}
\hfill
\begin{minipage}[t]{0.48\textwidth}
    \begin{items}
      \item [Nachteile]
      \item Kann nicht als einziges Lenksystem verwendet werden, oder nur mit erheblichen Zeitverzögerungen.
    \end{items}
\end{minipage}

\newpage
\subsection{Hindernisbewältigung}
\label{a2:Hindernisbewältigung}
Folgendes sind Ideen für die Hindernisbewältigung, jeweils mit Vor- und Nachteilen sowie einer Skizze oder Bildquelle. Die Aufgabe wurde in die Teilaufgaben Aufnahme und Rotation / Translation des Hindernisses unterteilt. 

\subsubsection{Aufnahme}
Dieses Kapitel beschreibt, wie das Hindernis aufgenommen werden soll.
\paragraph{Gabelstapler}
Das Prinzip `Gabelstapler` bedient sich der schon vorhandenen Löchern des Hindernisses.

\begin{figure}[h!]
        \centering
        \includegraphics[width=0.48\textwidth]{img/technologierecherche/Aufnahme/Gabelstapler.jpg}
        \caption{Prinzip angelehnt an einen Gabelstapler}
        \label{img:tech_Gaplerstapler}
\end{figure}

\begin{minipage}[t]{0.48\textwidth}
    \begin{items}
          \item [Vorteile]
          \item Einfache Griffkonstruktion.
          \item Das Hindernis ist in Richtung Boden sehr stabil fixiert.
    \end{items}
\end{minipage}
\hfill
\begin{minipage}[t]{0.48\textwidth}
    \begin{items}
    \item [Nachteile]
    \item Die Armkonstruktion muss eine hohe Präzision ermöglichen.
    \item Sensorik für den Arm nötig, Fahrzeug muss präzise gesteuert werden können.
    \item Löcher des Hindernisses müssen präzise erkannt werden.
    \item Es besteht die Gefahr des Herausfallens bei abruptem Bremsen.
    \end{items}
\end{minipage}
\newpage
%%%%%-----------------------------------------------
\paragraph{Vakuum-Sauggreifer}
Mit einem Vakuum-Sauggreifer kann das Hindernis durch Unterdruck angehoben werden \cite{vakuumgreifer}.

\begin{figure}[h!]
        \centering
        \includegraphics[width=0.48\textwidth]{img/technologierecherche/Aufnahme/Vakuumgreifer.jpg}
        \caption[Aufnahme über Vakuum-Sauggreifer]{Aufnahme über Vakuum-Sauggreifer \footnotemark}
        \label{img:tech_Vakuumgreifer}
\end{figure}
\footnotetext{Quelle: \url{https://www.rrg.de/elastomertechnik/sauggreifer/sauggreifer-vakuum/}}

\begin{minipage}[t]{0.48\textwidth}
    \begin{items}
          \item [Vorteile]
          \item Aussergewöhnliche Art, das Hindernis zu greifen.
    \end{items}
\end{minipage}
\hfill
\begin{minipage}[t]{0.48\textwidth}
    \begin{items}
        \item [Nachteile]
        \item Löcher des Hindernisses müssen präzise erkannt werden, damit diese nicht getroffen werden.
        \item Es ist unklar, wie viel Saugkraft auf der Oberfläche benötigt wird.
    \end{items}
\end{minipage}
\newpage
%%%%%-----------------------------------------------
\paragraph{Klemmen seitlich}
Das Hindernis wird seitlich eingeklemmt, um einen sicheren Halt zu erzeugen.

\begin{figure}[h!]
        \centering
        \includegraphics[width=0.48\textwidth]{img/technologierecherche/Aufnahme/Laengsweg_Griff.jpg}
        \caption{Klemme an der Seite des Hindernisses}
        \label{img:tech_Laengsweg_Griff}
\end{figure}

\begin{minipage}[t]{0.48\textwidth}
    \begin{items}
          \item [Vorteile]
          \item Ein Arm ist nicht erforderlich, und wenn doch, wird nur geringe Präzision benötigt.
          \item Sicherer Halt des Hindernisses.
    \end{items}
\end{minipage}
\hfill
\begin{minipage}[t]{0.48\textwidth}
    \begin{items}
          \item [Nachteile]
          \item Die Konstruktion eines Arms wäre komplex.
    \end{items}
\end{minipage}
\newpage
%%%%%-----------------------------------------------
\paragraph{Klemmen Breitenweg}
Das Hindernis wird durch Einklemmen über den Breitenweg fixiert. Dieses Prinzip liesse sich anpassen, um Berührungssensoren zu integrieren.

\begin{figure}[h]
        \centering
        \includegraphics[width=0.48\textwidth]{img/technologierecherche/Aufnahme/Breiterweg_Griff.jpg}
        \caption{Klemme über Breitenweg des Hindernisses}
        \label{img:tech_Breiterweg_Griff}
\end{figure}

\begin{minipage}[t]{0.48\textwidth}
    \begin{items}
          \item [Vorteile]
          \item Ein Arm ist nicht erforderlich, und wenn doch, wird nur geringe Präzision benötigt.
          \item Berührungssensoren können im Griff verbaut werden.
          \item Sicherer Halt des Hindernisses.
    \end{items}
\end{minipage}
\hfill
\begin{minipage}[t]{0.48\textwidth}
    \begin{items}
          \item [Nachteile]
          \item Beim Greifen ist unklar, wo sich anschliessend das Hindernis befindet.
          \item Die Konstruktion eines Arms wäre komplex.
    \end{items}
\end{minipage}
\newpage
%%%%%-----------------------------------------------
\paragraph{Mantis}
Das Hindernis wird seitlich gegriffen, wobei die Verbindung zum Zusammenstecken der Hindernisse genutzt wird. Dabei kommen zum Greifen elastische Materialien zum Einsatz.

\begin{figure}[h!]
        \centering
        \includegraphics[width=0.48\textwidth]{img/technologierecherche/Aufnahme/Mantis.jpg}
        \caption{Mantis Skizze}
        \label{img:tech_Mantis}
\end{figure}

\begin{minipage}[t]{0.48\textwidth}
    \begin{items}
          \item [Vorteile]
          \item Ein Arm ist nicht erforderlich, und wenn doch, wird nur geringe Präzision benötigt.
          \item Beim Greifen ist nur geringe Präzision der Sensoren erforderlich, da die Elastizität den Ausgleich übernimmt.
    \end{items}
\end{minipage}
\hfill
\begin{minipage}[t]{0.48\textwidth}
    \begin{items}
          \item [Nachteile]
          \item Das Hindernis könnte weggeschleudert werden.
          \item Es könnte problematisch sein, wenn die Klemmen nicht auf gleicher Höhe ansetzen.
    \end{items}
\end{minipage}
\newpage
\subsubsection{Rotation / Translation}
Dieses Kapitel beschreibt, wie das Hindernis nach der Aufnahme bewegt werden soll.

\paragraph{Bagger}
Die Konstruktion ist an einen Bagger angelehnt.

\begin{figure}[h!]
        \centering
        \includegraphics[width=0.48\textwidth]{img/technologierecherche/Rotation/kran.jpg}
        \caption[Bagger ähnliche Konstruktion]{Bagger ähnliche Konstruktion \footnotemark} 
        \label{img:tech_kran}
\end{figure}
\footnotetext{Quelle: \url{https://www.youtube.com/watch?v=VZRFHJfUkq4}}

\begin{minipage}[t]{0.48\textwidth}
    \begin{items}
          \item [Vorteile]
          \item Hohe Manövrierfähigkeit durch Aufbau gegeben.
          \item Kann mit unterschiedlichen Greifertypen kombiniert werden.
          \item Sensoren können auf der Oberseite des Fahrzeugs installiert werden.
    \end{items}
\end{minipage}
\hfill
\begin{minipage}[t]{0.48\textwidth}
    \begin{items}
          \item [Nachteile]
          \item Komplexer Aufbau.
    \end{items}
\end{minipage}
\newpage
%%--------------------------------------------------------
\paragraph{Rotation Fahrzeug}
Anstatt eines Armes, der sich dreht, dreht sich das ganze Fahrzeug mit dem Hindernis

\begin{figure}[h!]
        \centering
        \includegraphics[width=0.48\textwidth]{img/technologierecherche/Rotation/seitlich_mit_räder.jpg}
        \caption{Für die Rotation wird das ganze Fahrzeug gewendet}
        \label{img:tech_seitlich_mit_räder}
\end{figure}

\begin{minipage}[t]{0.48\textwidth}
    \begin{items}
          \item [Vorteile]
          \item Kann mit vielen Greifmethoden gebraucht werden.
          \item Sehr simple Konstruktion.
          \item Sensoren können auf der Oberseite des Fahrzeugs installiert werden.
    \end{items}
\end{minipage}
\hfill
\begin{minipage}[t]{0.48\textwidth}
    \begin{items}
          \item [Nachteile]
          \item Die Präzision ist abhängig von den Positionssensoren des Fahrzeugs.
          \item Die Steuerbarkeit hängt von der Manövrierfähigkeit des Fahrzeugs ab.
    \end{items}
\end{minipage}
\newpage
%%--------------------------------------------------------
\paragraph{Kran}
Diese Konstruktion ist an einen Kran angelegt und greift das Objekt von oben herab mit Greifkonstruktion "Klemme seitlich" oder Klemme Breitenweg".

\begin{figure}[H]
        \centering
        \includegraphics[width=0.7\textwidth]{img/technologierecherche/Rotation/seitlich_mit_rotation.jpg}
        \caption[Konstruktion an einen Kran angelehnt]{Konstruktion an einen Kran angelegt \footnotemark} 
        \label{img:tech_seitlich_mit_rotation}
\end{figure}
\footnotetext{Quelle: \url{https://www.youtube.com/watch?v=J7LGSNhFTU4}}

\begin{minipage}[t]{0.48\textwidth}
    \begin{items}
          \item [Vorteile]
           \item Sensoren können auf der Oberseite des Fahrzeugs installiert werden.
    \end{items}
\end{minipage}
\hfill
\begin{minipage}[t]{0.48\textwidth}
    \begin{items}
          \item [Nachteile]
          \item Komplexer Aufbau, wenn Arm kleine Korrekturen vornehmen soll.
          \item Die Steuerbarkeit hängt von der Manövrierfähigkeit des Fahrzeugs ab.
    \end{items}
\end{minipage}
\newpage
%%--------------------------------------------------------
\paragraph{Lagerung am Ende Arm}
Bei dieser Konstruktion bewegt ein Arm das aufgenommene Hindernis über das Fahrzeug. Damit das Hindernis auf der Rückseite wieder aufrecht steht, wird die Rotation beim Arm am Hindernis durchgeführt. 

\begin{figure}[H]
        \centering
        \includegraphics[width=0.48\textwidth]{img/technologierecherche/Rotation/ueberkopf_griff_gelagert.jpg}
        \caption{Skizze Rotation am Arm}
        \label{img:tech_ueberkopf_griff_gelagert}
\end{figure}

\begin{minipage}[t]{0.48\textwidth}
    \begin{items}
          \item [Vorteile]
          \item Durch fixe Armlänge kann Präzision gewährleistet werden.
    \end{items}
\end{minipage}
\hfill
\begin{minipage}[t]{0.48\textwidth}
    \begin{items}
          \item [Nachteile]
          \item Die über Kopfbewegung schränkt die Platzierung der Sensoren ein.
          \item Unklar wie viele Greifmethoden bei dieser Rotation verwendet werden können.
    \end{items}
\end{minipage}
\newpage
%%--------------------------------------------------------
\paragraph{Lagerung am Hindernis}
Zuerst wird das Hindernis in einer gelagerten Position ergriffen, sodass es während des Translationsvorgangs aufgrund der Schwerkraft aufrecht bleibt.

\begin{figure}[H]
        \centering
        \includegraphics[width=0.48\textwidth]{img/technologierecherche/Rotation/ueberkopf_objekt_gelagert.jpg}
        \caption{Skizze Lagerung am Hindernis} 
        \label{img:tech_ueberkopf_objekt_gelagert}
\end{figure}

\begin{minipage}[t]{0.48\textwidth}
    \begin{items}
          \item [Vorteile]
          \item Elegante Lösung, um das aufrecht Stehen des Hindernisses zu gewährleisten.
          \item Verschiedene Greifmethoden möglich.
    \end{items}
\end{minipage}
\hfill
\begin{minipage}[t]{0.48\textwidth}
    \begin{items}
          \item [Nachteile]
          \item Sensorplatzierung wird durch Hindernis-Transport über Fahrzeug hinweg eingeschränkt.
          
    \end{items}
\end{minipage}
\newpage
%%--------------------------------------------------------
\paragraph{Greifhaken}
Greifhaken-artige Konstruktion, welche die Löcher des Hindernisses ausnutzt.

\begin{figure}[h!]
        \centering
        \includegraphics[width=0.48\textwidth]{img/technologierecherche/Rotation/harponne.jpg}
        \caption{Skizze einer Greifhakenkonstruktion}
        \label{img:tech_harponne}
\end{figure}

\begin{minipage}[t]{0.48\textwidth}
    \begin{items}
          \item [Vorteile]
          \item Sehr einzigartige Methode.
    \end{items}
\end{minipage}
\hfill
\begin{minipage}[t]{0.48\textwidth}
    \begin{items}
          \item [Nachteile]
          \item Der Gabelstapler muss als Greifvorrichtung verwendet werden.
          \item Es ist unklar, ob es umsetzbar ist.
          \item Das Gehäuse muss ausreichend robust sein, da das Hindernis darüber gezogen werden kann.
          \item Sensorplatzierung wird durch Hindernis-Transport über Fahrzeug hinweg eingeschränkt.
    \end{items}
\end{minipage}
\newpage
%%--------------------------------------------------------
\subsection{Antrieb und Orientierung}
\label{a2:Antrieb_Orientierung}
Im folgenden Abschnitt werden die verschiedenen elektrischen Varianten – ''Antrieb'', ''Orientierung/Sensorik'' und ''Energiequelle'' – miteinander verglichen. Zu jeder Variante werden die Vor- und Nachteile aufgeführt, um die Auswahl der möglichen Lösungsansätze im morphologischen Kasten zu erleichtern.

\subsubsection{Antrieb}

In diesem Teil werden die verschiedenen elektrischen Antriebe aufgelistet und die jeweiligen Vor- und Nachteile genannt. 

\paragraph{DC-Motor}

Der DC-Motor ist ein Gleichstrom-Motor, der mit Bürsten am Kommutator arbeitet \cite{dc_motor}. 

\begin{minipage}[t]{0.48\textwidth}
\begin{items}
  \item [Vorteile]
  \item Einfache Drehzahlsteuerung
  \item Hohe Drehzahlen
  \item Hohes Drehmoment
\end{items}
\end{minipage}
\hfill
\begin{minipage}[t]{0.48\textwidth}
\begin{items}
  \item [Nachteile]
  \item EMV Störung durch Funkenbildung
  \item Schlechte Wärmeabführung
\end{items}
\end{minipage}

\paragraph{Brushless DC-Motor}

Brushless DC Motoren(BLDC) werden häufig im Modellbau oder Drohnen eingesetzt. Diese haben gegnüber normalen DC-Motoren keine Bürsten. Sie besitzen drei Wicklungen auf dem Stator und haben im Rotor einen Permanentmagneten. Die Spulen werden nun so angesteuert, dass es ein wechselndes Magnetfeld gibt. Dieses Drehfeld bringt den Rotor zur Drehung \cite{brushless}. 

\begin{minipage}[t]{0.48\textwidth}
\begin{items}
  \item [Vorteile]
  \item Hohe Effizienz
  \item Präzise Ansteuerung
  \item Hohe Lebensdauer
\end{items}
\end{minipage}
\hfill
\begin{minipage}[t]{0.48\textwidth}
\begin{items}
  \item [Nachteile]
  \item Komplexe Ansteuerung
  \item Hohe Kosten
\end{items}
\end{minipage}


\paragraph{Schrittmotor}

Der Schrittmotor wird vor allem bei genauen Positioniersystem wie Roboter, CNC oder auch 3D-Drucker eingesetzt. Diese Motoren haben viel mehr Wicklungen als ein BLDC. Dadurch können Sie kleinere Winkel anfahren und sind somit präzise Positionierungsantriebe \cite{schrittmotor}. 

\begin{minipage}[t]{0.48\textwidth}
\begin{items}
  \item [Vorteile]
  \item Präzise Ansteuerung
  \item Hohes Drehmoment bei tiefen Drehzahlen
  \item Kein Encoder nötig (bei genügender Auslegung)
\end{items}
\end{minipage}
\hfill
\begin{minipage}[t]{0.48\textwidth}
\begin{items}
  \item [Nachteile]
  \item Reduzierte Leistung bei hoher Geschwindigkeit
  \item Hoher Stromverbrauch
  \item Erzeugen Vibrationen bei hohen Drehzahlen
\end{items}
\end{minipage}

\subsubsection{Orientierung / Sensorik}

In diesem Abschnitt wird genauer betrachtet, wie das Fahrzeug die Umwelt wahrnimmt. Dabei werden wieder die einzelnen Vor- und Nachteile aufgelistet.

\paragraph{Liniensensor}
Ein Liniensensor funktioniert mit mehreren Fotodioden und LEDs. Die LEDs beleuchten den Boden und die in einer Reihe angeordneten Fotodioden nehmen mehr oder weniger von diesem reflektierten Licht auf. Je nach Untergrund wird es mehr oder weniger reflektiert. So kann man die Linie erkennen und dieser folgen \cite{liniensensor}.
 

\begin{minipage}[t]{0.48\textwidth}
\begin{items}
  \item [Vorteile]
  \item Schnelle Reaktionszeit
  \item Einfacher Aufbau
\end{items}
\end{minipage}
\hfill
\begin{minipage}[t]{0.48\textwidth}
\begin{items}
  \item [Nachteile]
  \item Hoher Kontrast nötig
  \item Störungsanfällig auf Fremdlicht
  \item Kalibrierung notwendig
\end{items}
\end{minipage}

\paragraph{Infrarot}
Mit Infrarot (IR) Sensoren können Objekte erkannt und die Distanz zu ihnen ermittelt werden \cite{ir_sensor}.


\begin{minipage}[t]{0.48\textwidth}
\begin{items}
  \item [Vorteile]
  \item Schnelle Reaktionszeit
  \item Hohe Empfindlichkeit
  \item Robust und langlebig
\end{items}
\end{minipage}
\hfill
\begin{minipage}[t]{0.48\textwidth}
\begin{items}
  \item [Nachteile]
  \item Begrenzte Reichweite
  \item Abhängigkeit Oberflächenbeschaffenheit
  \item Störungen durch andere IR-Quellen
\end{items}
\end{minipage}

\paragraph{Ultraschall}
Ein Ultraschallsensor erkennt Objekte und misst deren Entfernung, indem er Schallwellen aussendet und deren Reflexion auswertet \cite{ultraschall}.

\begin{minipage}[t]{0.48\textwidth}
\begin{items}
  \item [Vorteile]
  \item Unabhängig vom Material
  \item Hohe Präzision
  \item Robust gegenüber Umwelteinflüssen
\end{items}
\end{minipage}
\hfill
\begin{minipage}[t]{0.48\textwidth}
\begin{items}
  \item [Nachteile]
  \item Tote Zone in der Nähe des Sensors
  \item Unpräzise bei komplexer Umgebung
  \item Kleine Objekte schwer detektierbar
\end{items}
\end{minipage}

\paragraph{Encoder}
Ein Encoder detektiert die Umdrehungen eines Motors und wandelt diese in ein digitales Signal um. Mit diesem Signal kann die Steuerung den Motor präzise ansteuern \cite{encoder}.
 

\begin{minipage}[t]{0.48\textwidth}
\begin{items}
  \item [Vorteile]
  \item Hohe Präzision
  \item Schnelle Reaktionszeit
\end{items}
\end{minipage}
\hfill
\begin{minipage}[t]{0.48\textwidth}
\begin{items}
  \item [Nachteile]
  \item Hohe Kosten
  \item Anfällig auf Schmutz und Elektromagnetische Verträglichkeit (EMV)
\end{items}
\end{minipage}


\subsubsection{Energiequelle}

In diesem Abschnitt werden verschiedene Energiequellen untersucht.

\paragraph{Lithium-Polymer}
Der Lithium-Polymer (LiPo) Akku ist ein formbarer Akku, der vor allem im Modellbau und bei Drohnen eingesetzt wird \cite{lipo}.
 

\begin{minipage}[t]{0.48\textwidth}
\begin{items}
  \item [Vorteile]
  \item Leicht und flexibel in der Formgebung
  \item Hohe Leistungsdichte
  \item Hohe Entladerate
\end{items}
\end{minipage}
\hfill
\begin{minipage}[t]{0.48\textwidth}
\begin{items}
  \item [Nachteile]
  \item Empfindlichkeit zu Über- und Tiefenentladung
  \item Empfindlich auf mechanische Schläge
  \item Sicherheit in Bezug auf Feuergefahr
\end{items}
\end{minipage}


\paragraph{Lithium-Ionen}

Lithium-Ionen-Akkumulatoren (Li-Ion-Akkus) stellen eine zuverlässige Energiespeichertechnologie dar, die häufig in Laptops und Smartphones eingesetzt wird. Ein wesentlicher Vorteil dieser Akkutypen ist das Fehlen des sogenannten Memory-Effekts. Dieser Effekt beschreibt das Phänomen, bei dem ein Akkumulator, der vor dem erneuten Aufladen nicht vollständig entladen wurde, diesen Ladezustand als tiefsten Entladezustand ``speichert``, was zu einer Reduktion der nutzbaren Kapazität führen kann \cite{li-ion}.

\begin{minipage}[t]{0.48\textwidth}
\begin{items}
  \item [Vorteile]
  \item Lange Lebensdauer
  \item Hohe Energiedichte
  \item Kein Memory-Effekt
\end{items}
\end{minipage}
\hfill
\begin{minipage}[t]{0.48\textwidth}
\begin{items}
  \item [Nachteile]
  \item Temperaturempfindlich
  \item Hohe Kosten
  \item Sicherheit in Bezug auf Feuergefahr
\end{items}
\end{minipage}
 

\paragraph{Nickel-Metallhydrid}

Nickel-Metallhydrid (NiMH) Akkus sind in vielen Haushaltsgeräten, Digitalkameras oder Taschenlampen zu finden \cite{nimh}.

\begin{minipage}[t]{0.48\textwidth}
\begin{items}
  \item [Vorteile]
  \item Moderate Energiedichte
  \item Geringer Memory Effekt
  \item Sicher
\end{items}
\end{minipage}
\hfill
\begin{minipage}[t]{0.48\textwidth}
\begin{items}
  \item [Nachteile]
  \item Selbstentladung
  \item Kapazitätsverlust
  \item Temperaturempfindlich
\end{items}
\end{minipage}

\end{document}