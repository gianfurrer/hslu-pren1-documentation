\newpage
\section{Konzeptfindung}

Unser Endziel ist ein möglichst einfaches und zuverlässiges Gesamtkonzept, welches möglichst robust und Fehler-unanfällig funktioniert. Dies möchten wir erreichen, in dem wir vorgängig alle Lösungsansätze aus der Technologierecherche, welche mit grossen Risiken verbunden sind, erkennen und ausscheiden. Mit den verbleibenden Ansätzen wird ein Morphologischer-Kasten erstellt, um ein Gesamtkonzept zu entwickeln.

\subsection{Vorauswahl}
    \subsubsection{Fortbewegung}
        Da das Fahrzeug in der Lage sein muss, sich an Ort und Stelle um die eigene Achse zu wenden, sind die Optionen Knicklenkung, Achsschenkellenkung und Drehschemellenkung nicht geeignet.
        Das Prinzip <<Fahrzeug ab bocken und drehen>> ist als alleiniges Lenksystem ungeeignet, weil es zu viel Zeit benötigt für das Lenkmanöver. Es wird als optionalen Plan-B beibehalten für den Fall, dass das Fahrzeug beim Wenden während dem Hindernishandling verrutscht. Falls dieses Problem beim Testen auftritt, würde das System nachgerüstet werden.
    
    \subsubsection{Hindernis Bewältigung}
        \paragraph{Aufnahme}
            Da das Konzept möglichst einfach und zuverlässig gestaltet werden soll, scheiden die Ideen "Gabelstapler" und "Vakuumgreifer" aus der Vorauswahl aus. Der Gabelstapler erfordert eine äusserst präzise und aufwendige Sensorik, um die Löcher für die Aufnahme genau zu treffen. Der Vakuumgreifer wiederum zeigt im Vergleich zu den "Klemmen" eine geringere Zuverlässigkeit. Um eine vergleichbare Zuverlässigkeit zu erreichen, wären umfangreiche Tests erforderlich, ohne dass der Vakuumgreifer dabei signifikante Vorteile gegenüber den 'Klemmen' bieten würde.

        \paragraph{}
            
        \paragraph{Rotation und Translation}
        Weil ein beweglicher arm die Komplexität automatisch nicht nur mechniasch 

\subsubsection{Elektrotechnik}
    \paragraph{}

\subsection{Objekterkennung}



\subsubsection{Simulator}
    \paragraph{}

\subsection{Morphologischer Kasten}
        