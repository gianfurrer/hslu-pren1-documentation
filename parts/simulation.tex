\newpage
\section{Simulator}

Der Simulator dient dazu, die Funktionalität der Software des autonomen Fahrzeugs zu testen, bevor der physische Prototyp6   gebaut wird.

\subsection{Spezifikation}

In diesem Abschnitt wird definiert, was der Simulator genau leisten soll.

\begin{itemize}
    \item Aufzeigen, dass (ohne Beachtung der Hindernisse) der schnellste Weg ins Ziel gefunden wird.
    \item 
\end{itemize}

\subsubsection{Nutzwertanalyse}
Die Wegfindung und Entscheidungsprozesse des Fahrzeugs sollen anhand eines Simulators und vor dessen Bau getestet werden können. Bei der Technologierecherche wurden zwei Methoden recherchiert, die in diesem Abschnitt genauer analysiert werden.
Der passendste Simulator wird anhand einer Nutzwertanalyse mit den folgenden Kriterien gewählt.

\begin{itemize}
\item \textbf{Anpassbarkeit}: Das wichtigste Kriterium ist, dass der Simulator genau an die Problemstellung angepasst werden kann.
\item \textbf{Geschwindigkeit}: Die Simulation soll auf handelsüblichen Rechnern mit moderater Leistung schnell laufen, um einfache Softwareiteration zu ermöglichen.  
\item \textbf{Komplexität}: Das Projekt ist zeitlich limitiert. Der Aufwand für die Implementierung und Anpassung des Simulators muss berücksichtigt werden.
\item \textbf{Schnittstellen}: Damit der Simulator angesteuert werden kann, soll er möglichst einfache Schnittstellen besitzen.
\end{itemize}


\begin{table}[H]
\resizebox{\textwidth}{!}{

 \centering
 \renewcommand{\arraystretch}{1.5}
     \begin{tabular}{|c|c|c|c|c|c|c|c|c|c|}
        \hline
        \multicolumn{2}{|c|}{} & 
        \multicolumn{2}{c|}{\textbf{Selbst entwickelt}} &
        \multicolumn{2}{c|}{\textbf{MicroMouseSimulator}}\\ 
        \hline
        
        \textbf{Kriterium} & \textbf{Gewicht} &
        \textbf{Bewertung} & \textbf{Punkte} &
        \textbf{Bewertung} & \textbf{Punkte}\\
        \hline

        Anpassbarkeit & 40 &
        10 & 360 &
        5 & 200 \\
        \hline

        Schnittstellen & 30 &
        10 & 300 &
        8 & 240  \\
        \hline

        Komplexität & 20 &
        8 & 160 &
        4 & 80 \\
        \hline

        Geschwindigkeit & 10 &
        8 & 80 &
        9 & 90 \\
        \hline


        
        \textbf{Nutzwert:} & 100 & & 900 & & 610\\ \hline
    \end{tabular}
 }
    \caption{Nutzwertanalyse Simulator}
    \label{tab:nutzwertanalyse_simulator}
\end{table}

Ein selbstgebauter Simulator erreicht in der durchgeführten Nutzwertanalyse (siehe Tabelle \ref{tab:nutzwertanalyse_simulator}) die beste Gesamtbewertung. Das hohe Mass an Anpassbarkeit und die Möglichkeit, Schnittstellen nach Belieben zu implementieren, sprechen klar für eine Eigenentwicklung. Der MicroMouseSimulator ist bereits auf Geschwindigkeit optimiert, jedoch sollte der Unterschied auf modernen Rechnern einen zu kleinen Unterschied machen, als dass die anderen Nachteile aufgeholt würden. Statt sich in den Source-Code eines bereits bestehenden Projektes einzulesen, kann von Grund auf für die vorhandene Problemstellung entwickelt werden.