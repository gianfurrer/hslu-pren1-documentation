\newpage
\section{Technologierecherche}

Im folgenden Kapitel wird zuerst ein grober Überblick über die Technologierecherche gegeben. Im Anschluss folgt zu jedem Spezialgebiet, (Informatik, Maschinenbau, Elektrotechnik), eine kurze Zusammenfassung über die Resultate der Recherche. Das Projekt wird in kleinere Teilfunktionen gespalten. Diese sind im folgenden Aufgelistet:
\begin{itemize}
    \item Fortbewegung
    \item Antrieb
    \item Steuerung
    \item Orientierung
    \item Objekterkennung
    \item Wegfindung
    \item Sicherheit
    \item Energiequelle
    \item Hindernis Bewältigung
\end{itemize}


\subsection{Überblick}
In der folgenden Tabelle sind zu jeder Teilfunktion verschiede Themen aufgelistet, die eine Bewertung und Link besitzen. Die Bewertung ist dazu da, wie relevant das Thema für die Teilfunktion ist.

\scriptsize
\begin{longtable}{l@{\extracolsep{\fill}}p{2cm}p{2cm}p{4cm}p{1.5cm}lll}
%\toprule
\textbf{Dep.} & \textbf{Teilfunktion} & \textbf{Thema} &
\textbf{Beschreibung} & \textbf{Bewertung (0-10)} & \textbf{Quelle} & \textbf{Abfragedatum} &
\textbf{Wer}\tabularnewline
%\midrule
\endhead

M & Fortbewegung & Beweglichkeit & 2-Rad gegensinnig für 360° \newline Drehung im Stand & 8 & MC-Car & 01.10.2024 & Joel
\tabularnewline
M & Fortbewegung & Beweglichkeit & 4-Rad Differential steering / \newline skid steer & 8 & \href{https://en.wikipedia.org/wiki/Differential_steering}{Link} /\href{https://science.howstuffworks.com/transport/engines-equipment/skid-steer2.htm}{Link} & 03.10.2024 & Silas
\tabularnewline
M & Fortbewegung & Beweglichkeit & Omniwheels & 9 & \href{https://de.wikipedia.org/wiki/Allseitenrad}{Link} / \href{https://www.youtube.com/watch?v=wwQQnSWqB7A}{Link} & 03.10.2024 & Silas
\tabularnewline
M & Fortbewegung & Beweglichkeit & Knicklenkung & 4 & \href{https://de.wikipedia.org/wiki/Knicklenkung}{Link} & 03.10.2024 & Silas
\tabularnewline
M & Fortbewegung & Beweglichkeit & Achsschenkellenkung & 5 & \href{https://de.wikipedia.org/wiki/Achsschenkel#:~:text=Die%20Erfindung%20der%20Achsschenkellenkung%20bedeutete,wird%20bei%20Automobilen%20ausschlie%C3%9Flich%20verwendet.}{Link} & 03.10.2024 & Silas
\tabularnewline
M & Fortbewegung & Beweglichkeit & Drehschemellenkung & 5 & \href{https://www.staplerberater.de/auswahlkriterien/lenkungsarten}{Link} & 03.10.2024 & Silas
\tabularnewline
M & Fortbewegung & Beweglichkeit & Fahrzeug abbocken und an Stelle auf Drehkranz wenden & 6 & \href{https://www.kaiserkraft.ch/hubgeraete/hub-und-verladetische/auto-niveaugeraet-mit-drehscheibe/drehscheiben-1110-mm/p/M1142876/?articleNumber=118558&lang=de_CH&customerType=B2C&lang=&infinity=ict2~net~gaw~cmp~PM_DE-shopping24-Jarvis-0~ag~~ar~~kw~~mt~&gad_source=1&gclid=CjwKCAjwgfm3BhBeEiwAFfxrGxsQhJoEWwY3dNM_OYKFg2NOgoHXLP2OeyLmOZFTVnzHt7PvNpgCbhoCACQQAvD_BwE}{Link} & 03.10.2024 & Silas
\tabularnewline

E & Antrieb & Motor & Schrittmotor & 8 & \href{https://wiki.bu.ost.ch/infoportal/_media/hardware/sysp/bauteile/schrittmotor_kurz_erklaert_d.pdf}{Link} & 27.09.2024 & Thomas
\tabularnewline
E & Antrieb & Motor & DC Motor & 7 & \href{https://www.arrow.de/research-and-events/articles/which-dc-motor-is-best-for-your-application}{Link} & 27.09.2024 & Thomas
\tabularnewline
E & Antrieb & Motor & Brushless & 7 & \href{https://www.renesas.com/en/support/engineer-school/brushless-dc-motor-01-overview}{Link} & 27.09.2024 & Thomas
\tabularnewline

I & Steuerung & Hardware & Raspberry Pi & 8 & \href{https://www.raspberrypi.com/documentation/computers/raspberry-pi.html}{Link} & 29.09.2024 & Thomas
\tabularnewline
I & Steuerung & Hardware & Arduino & 5 & \href{https://arduino.cc/en/hardware#boards-1}{Link} & 29.09.2024 & Thomas
\tabularnewline
I & Steuerung & Hardware & Tiny & 8 & HSLU & 01.10.2024 & Joel
\tabularnewline
I & Steuerung & Hardware & ESP32 & 7 & \href{https://www.espressif.com/en/products/devkits/esp32-devkitc}{Link} & 03.10.2024 & Thomas
\tabularnewline
E & Orientierung & Sensorik & Linienfolger & 5 & \href{https://pglu.ch/3-mit-fahrroboter-linie-folgen/?srsltid=AfmBOor3qIbdXGD1WYtV-YadIVjE2Urm7U3QGtes_IjcCzCVViC2yody}{Link} & 04.10.2024 & Joel
\tabularnewline
E & Orientierung & Sensorik & Linienfolger & 5 & \href{https://spacehal.github.io/docs/robotik/edgeFollower}{Link} & 04.10.2024 & Joel
\tabularnewline
E & Orientierung & Sensorik & Infrarot & 5 & \href{https://www.elektronik-kompendium.de/sites/raspberry-pi/2802011.htm}{Link} & 27.09.2024 & Thomas
\tabularnewline
E & Orientierung & Sensorik & Ultraschall & 5 & \href{https://elektro.turanis.de/html/prj121/index.html}{Link} & 27.09.2024 & Thomas 
\tabularnewline
E & Orientierung & Sensorik & Encoder & 7 & \href{https://www.arrow.de/research-and-events/articles/rotary-encoders-how-to-pair-with-an-arduino-board}{Link} & 29.09.2024 & Thomas
\tabularnewline
E & Orientierung & Sensorik & Optischer Sensor & 5 & \href{https://global.sharp/products/device/lineup/data/pdf/datasheet/gp2y0e02a_e.pdf}{Link} & 01.10.2024 & Joel
\tabularnewline

I & Objekterkennung & Deep Learning & Deep Learning Frameworks & 7 &  \href{https://www.simplilearn.com/tutorials/deep-learning-tutorial/deep-learning-frameworks} {Link}&  27.09.2024 & Gian
\tabularnewline
I & Objekterkennung & Deep Learning & TensorFlow in 100 seconds & 10 &
\href{https://www.youtube.com/watch?v=i8NETqtGHms}{Link} & 29.09.2024 & Gian
\tabularnewline
I & Objekterkennung & Deep Learning & What is Object Detection? & 9 &
\href{https://www.ibm.com/topics/object-detection#:~:text=Object%20detection%20is%20a%20technique,imaging%20to%20self%2Ddriving%20cars.}{Link}
& 02.10.2024 & Gian
\tabularnewline
I & Objekterkennung & Deep Learning & Region Based CNN - Wikipedia & 6 &
\href{https://en.wikipedia.org/wiki/Region_Based_Convolutional_Neural_Networks}{Link}
& 02.10.2024 & Gian
\tabularnewline

I & Wegfindung & Algorithmus & Informationen über Wegfindung & 8 &
\href{https://de.wikipedia.org/wiki/Pathfinding}{Link} & 02.10.2024 & Gian
\tabularnewline
I & Wegfindung & Algorithmus & Comparing Dijkstra’s and A* Search Algorithm & 7 &
\href{https://medium.com/@miguell.m/dijkstras-and-a-search-algorithm-2e67029d7749}{Link}
& 02.10.2024 & Gian
\tabularnewline
I & Wegfindung & Algorithmus & Rapidly exploring random tree & 4 &
\href{https://en.wikipedia.org/wiki/Rapidly_exploring_random_tree}{Link}
& 03.10.2024 & Gian
\tabularnewline
I & Wegfindung & Algorithmus & D\* & 9 &
\href{https://en.wikipedia.org/wiki/D*}{Link} & 03.10.2024 & Gian
\tabularnewline
I & Wegfindung & Algorithmus & Goal directed shortest path queries using precomputed cluster distances & 8 &
\href{https://publikationen.bibliothek.kit.edu/1000009512}{Link} & 04.10.2024 & Patrick
\tabularnewline
I & Wegfindung & Algorithmus & Floyd Warshall shortest path algorithm & 8 &
\href{https://en.wikipedia.org/wiki/Floyd%E2%80%93Warshall_algorithm}{Link} & 04.10.2024 & Patrick
\tabularnewline
I & Wegfindung & Algorithmus & Depth First Search (DFS) & 7 &
\href{https://en.wikipedia.org/wiki/Depth-first_search}{Link} & 04.10.2024 & Patrick
\tabularnewline
I & Wegfindung & Algorithmus & A decremental approach with the \* algorithm & 4 &
\href{https://doi.org/10.1016/j.measurement.2014.10.014}{Link} & 04.10.2024 & Patrick
\tabularnewline

E & Sicherheit & Not-Aus & Implementierung eines Not-Aus & 6 & \href{https://www.eaton.com/ie/en-gb/markets/machine-building/service-and-support-machine-building-moem-service-eaton/blogs/emergency-stop-circuit---blogs---eaton.html}{Link} & 27.09.2024 & Thomas
\tabularnewline
E & Sicherheit & Sicherheit & mobiler Roboter & 5 & \href{https://tuprints.ulb.tu-darmstadt.de/18674/1/10.1524_auto.51.10.435.19576.pdf}{Link} & 27.09.2024 & Thomas 
\tabularnewline

E & Energiequelle & Akkumulatoren & LiPo & 8 & \href{https://www.lion-care.com/lipo-akkus-eigenschaften-vorteile-und-mehr}{Link} & 27.09.2024 & Thomas
\tabularnewline
E & Energiequelle & Akkumulatoren & Li-Ion & 7 & \href{https://poleenergy.ch/shop_content.php?coID=32}{Link} & 27.09.2024 & Thomas
\tabularnewline
E & Energiequelle & Solarpanel & Run Arduino Offgrid & 6 & \href{https://voltaicsystems.com/solar-arduino-guide/}{Link} & 29.09.2024 & Thomas
\tabularnewline
E & Energiequelle & Akkumulatoren & Nickel-Metallhydrid-Akkus \newline (Batterie) & 7 & \href{https://voltaicsystems.com/solar-arduino-guide/}{Link} & 29.09.2024 & Joel
\tabularnewline

M & Hindernisbewältigung & Aufnahme & gegen Fahrzeug pressen & 8 &  & 03.10.2024 & Silas
\tabularnewline
M & Hindernisbewältigung & Aufnahme & Palettengabel in Löcher von Hindernis  & 5 &  & 03.10.2024 & Silas
\tabularnewline
M & Hindernisbewältigung & Aufnahme & Vakuumgreifer & 5 & \href{https://www.schmalz.com/de-ch/glossar/vakuumgreifer/}{Link} & 03.10.2024 & Silas
\tabularnewline
M & Hindernisbewältigung & Transport & ganzes Fahrzeug wenden & 8 &  & 03.10.2024 & Silas
\tabularnewline
M & Hindernisbewältigung & Transport & über Fahrzeug hinweg mit Ausleger & 6 &  & 03.10.2024 & Silas
\tabularnewline
M & Hindernisbewältigung & Transport & über Fahrzeug hinweg mit Förderband & 3 &  & 03.10.2024 & Silas
\tabularnewline
M & Hindernisbewältigung & Aufnahme & Vakuumgreifer & 8 & \href{https://www.youtube.com/shorts/alxwWgzSVss}{Link}& 03.10.2024 & Silvan
\tabularnewline
M & Hindernisbewältigung & Rotation & Kran & 9 & \href{https://www.youtube.com/watch?v=VZRFHJfUkq4&feature=youtu.be}{Link}& 03.10.2024 & Silvan
\tabularnewline
M & Hindernisbewältigung & Rotation & Kran von oben & 8 & \href{https://www.youtube.com/watch?v=J7LGSNhFTU4}{Link}& 03.10.2024 & Silvan
\tabularnewline
\caption{Technologierecherche Übersicht}
\label{tab:technologierecherche}
\end{longtable}
\normalsize

\newpage
\subsection{Informatik}

Dieser Abschnitt enthält detaillierte Rechercheergebnisse über die Bereiche der Informatik. Dabei werden Vor- und Nachteile aufgezeigt und die einzelnen Algorithmen erklärt.

\subsubsection{Wegfindung}

Die Topologie des Weges mit Start und Zielmöglichkeiten ist bereits bekannt.
Somit kann anhand eines Wegfindungsalgorithmus der schnellste Weg berechnet werden.

\paragraph{Dijkstra}

Findet den kürzesten Weg von einem gegebenen Startknoten zu allen anderen Knoten in einem Graphen mit nicht-negativen Gewichten.

\begin{minipage}[t]{0.48\textwidth}
\begin{items}
  \item [Vorteile]
  \item Optimierter Greedy-Algorithmus, der immer die optimale Lösung findet.
  \item Funktioniert gut in dichten Graphen, wo die Anzahl der Kanten hoch ist.
\end{items}
\end{minipage}
\hfill
\begin{minipage}[t]{0.48\textwidth}
\begin{items}
  \item [Nachteile]
  \item Nicht zielgerichtet, kann zu Zeitverschwendung führen.
  \item Erfordert die Speicherung aller Knoten im Speicher, zu hohem Speicherbedarf führen kann.
  \item Nicht dynamisch; Änderungen am Graphen erfordern einen vollständigen Neudurchlauf des Algorithmus.
\end{items}
\end{minipage}
\paragraph{A*-Algorithmus}

Ein heuristischer Algorithmus, der den kürzesten Pfad von einem Startknoten zu einem Zielknoten findet. Er kombiniert Dijkstra mit einer Heuristik, um effizientere Entscheidungen zu treffen.

\begin{minipage}[t]{0.48\textwidth}
\begin{items}
  \item [Vorteile]
  \item Findet optimale Lösungen und ist oft schneller als Dijkstra, besonders bei großen Graphen.
  \item Anpassbar durch Wahl der Heuristik.
\end{items}
\end{minipage}
\hfill
\begin{minipage}[t]{0.48\textwidth}
\begin{items}
  \item [Nachteile]
  \item Die Effizienz hängt stark von der Wahl der Heuristik ab.
  \item Kann in bestimmten Fällen langsamer sein als Dijkstra, wenn die Heuristik schlecht gewählt ist.
\end{items}
\end{minipage}

\paragraph{Rapidly-exploring Random Tree (RRT)}

Ein probabilistischer Algorithmus, der vor allem in der Robotik eingesetzt wird,
um schnell einen Pfad in einem komplexen Raum zu finden.

\begin{minipage}[t]{0.48\textwidth}
\begin{items}
  \item [Vorteile]
  \item Sehr effektiv in hochdimensionalen Räumen.
  \item Flexibel, da er dynamisch auf neue Hindernisse reagieren kann.
\end{items}
\end{minipage}
\hfill
\begin{minipage}[t]{0.48\textwidth}
\begin{items}
  \item [Nachteile]
  \item Kann suboptimale Lösungen liefern, da der Pfad eher zufällig als deterministisch erzeugt wird.
  \item Benötigt oft zusätzliche Post-Processing-Schritte zur Glättung des Pfades.
\end{items}
\end{minipage}

\paragraph{Dynamic A* (D* Lite)}

Eine Erweiterung des A*-Algorithmus, die speziell für dynamische Umgebungen entwickelt wurde, in denen sich Hindernisse während der Laufzeit ändern können.

\begin{minipage}[t]{0.48\textwidth}
\begin{items}
  \item [Vorteile]
  \item Effiziente Anpassung des Pfades bei Änderungen der Umgebung.
  \item Minimiert die Berechnungen, indem nur die betroffenen Teile des Pfades neu berechnet werden.
\end{items}
\end{minipage}
\hfill
\begin{minipage}[t]{0.48\textwidth}
\begin{items}
  \item [Nachteile]
  \item Kann in sehr dynamischen Umgebungen langsamer sein, wenn viele Änderungen gleichzeitig auftreten.
  \item Komplexität der Implementierung im Vergleich zu einfacheren Algorithmen.
\end{items}
\end{minipage}


\paragraph{Goal directed shortest path queries using precomputed cluster distances}
Beschleunigte Version von Dijkstra's Algorithmus, wobei für Graphen-Cluster-Distanzen vorberechnet werden.

\begin{minipage}[t]{0.48\textwidth}
\begin{items}
  \item [Vorteile]
  \item Vorgenerierte kürzeste Pfade erlauben schnelle Wechsel bei Graph-Anpassungen
  \item Schnell bei kleinen Graphen
\end{items}
\end{minipage}
\hfill
\begin{minipage}[t]{0.48\textwidth}
\begin{items}
  \item [Nachteile]
  \item Relativ grosser Speicheraufwand für die vorberechneten kürzesten Pfade.
\end{items}
\end{minipage}

\paragraph{Floyd Warshall shortest path algorithm}
Findet den kürzesten Weg von allen Knoten zu allen anderen Knoten in einem Graphen.

\begin{minipage}[t]{0.48\textwidth}
\begin{items}
  \item [Vorteile]
  \item Effizient für kleine und dichte Graphen
\end{items}
\end{minipage}
\hfill
\begin{minipage}[t]{0.48\textwidth}
\begin{items}
  \item [Nachteile]
  \item Langsamer bei grossen Graphen als Dijkstra oder A* 
\end{items}
\end{minipage}

\paragraph{Tiefensuche DFS}
Sucht den Graphen nach einem Objekt ab. Dabei werden neu gefundene Wege weiterverfolgt, bevor weitere Nachbarn von Knoten besucht werden.

\begin{minipage}[t]{0.48\textwidth}
\begin{items}
  \item [Vorteile]
  \item Effiziente Suche, wenn Graph nicht von Beginn an bekannt ist
\end{items}
\end{minipage}
\hfill
\begin{minipage}[t]{0.48\textwidth}
\begin{items}
  \item [Nachteile]
  \item Kann sich in Sackgassen des Graphen lange aufhalten.
\end{items}
\end{minipage}

\subsubsection{Objekterkennung}

Der Weg, die Hindernisse sowie die Pylonen müssen anhand von Software erkannt und kategorisiert werden. Um die beiden Objekte voneinander zu unterscheiden, müssen die Objekte anhand von Bildern erkannt werden.

\paragraph{Convolutional Neural Networks (CNN)}

Ein tiefes Lernverfahren, das besonders für die Bildverarbeitung und Objekterkennung geeignet ist. CNNs nutzen Faltungsschichten, um Merkmale aus Bildern zu extrahieren.

\begin{minipage}[t]{0.48\textwidth}
\begin{items}
  \item [Vorteile]
  \item Hohe Genauigkeit bei der Objekterkennung, besonders bei komplexen Szenen.
  \item Automatische Merkmalserkennung ohne manuelle Feature-Engineering.
\end{items}
\end{minipage}
\hfill
\begin{minipage}[t]{0.48\textwidth}
\begin{items}
  \item [Nachteile]
  \item Hoher Rechenaufwand und benötigte Datenmengen für das Training.
  \item Empfindlich gegenüber Veränderungen in der Beleuchtung und Bildqualität.
\end{items}
\end{minipage}

\paragraph{YOLO (You Only Look Once)}

Ein Echtzeit-Objekterkennungsalgorithmus, der in der Lage ist, mehrere Objekte in einem Bild in einem einzigen Durchgang zu erkennen. Basierend auf CNN.

\begin{minipage}[t]{0.48\textwidth}
\begin{items}
  \item [Vorteile]
  \item Sehr schnell und effizient, ideal für Echtzeitanwendungen.
  \item Kann mehrere Klassen in einem Bild gleichzeitig erkennen.
\end{items}
\end{minipage}
\hfill
\begin{minipage}[t]{0.48\textwidth}
\begin{items}
  \item [Nachteile]
  \item Geringere Genauigkeit bei kleineren Objekten im Vergleich zu anderen Methoden.
  \item Schwierigkeiten bei der Erkennung von überlappenden Objekten.
\end{items}
\end{minipage}

\paragraph{Haar-Cascade-Klassifikatoren}

Ein Algorithmus zur schnellen Objekterkennung, der häufig in Echtzeitanwendungen verwendet wird.

\begin{minipage}[t]{0.48\textwidth}
\begin{items}
  \item [Vorteile]
  \item Schnell und effizient für die Erkennung bestimmter Objekte wie Gesichter oder Pylonen.
  \item Relativ einfach zu implementieren und zu trainieren.
\end{items}
\end{minipage}
\hfill
\begin{minipage}[t]{0.48\textwidth}
\begin{items}
  \item [Nachteile]
  \item Nicht so robust gegen Variationen in Beleuchtung und Hintergrund.
  \item Begrenzte Genauigkeit und Flexibilität im Vergleich zu tiefen Lernansätzen.
\end{items}
\end{minipage}

\paragraph{Region-based CNN (R-CNN)}

Ein Ansatz, der die Genauigkeit der Objekterkennung verbessert, indem er Regionenvorschläge nutzt und diese dann mit einem CNN klassifiziert.

\begin{minipage}[t]{0.48\textwidth}
    \begin{items}
      \item [Vorteile]
      \item Hohe Genauigkeit bei der Erkennung von Objekten in Bildern.
      \item Gut geeignet für die Erkennung in komplexen Szenen.
    \end{items}
\end{minipage}
\hfill
\begin{minipage}[t]{0.48\textwidth}
    \begin{items}
      \item [Nachteile]
      \item Hoher Rechenaufwand, was die Echtzeitfähigkeit einschränkt.
      \item Erfordert eine sorgfältige Abstimmung der Hyperparameter und Regionenvorschläge.
    \end{items}
\end{minipage}



\subsubsection{Simulation}


\newpage
\subsection{Hindernis Bewältigung}
Folgendes sind Ideen für die Hindernisbewältigung, jeweils mit einer Skizze oder Quelle. Die Aufgabe wurde in den Teil Aufnahme des Hindernisses und Rotation / Transport unterteilt.
\subsubsection{Aufnahme Hindernis}
\begin{figure}[h!]
    \centering
    \begin{minipage}{0.45\textwidth}
        \centering
        \includegraphics[width=\textwidth]{img/technologierecherche/Aufnahme/Gabelstapler.jpg}
        \caption{Prinzip angelehnt an einen Gabelstapler}
        \label{img:tech_Gaplerstapler}
    \end{minipage}
    \hfill
    \begin{minipage}{0.45\textwidth}
        \centering
        \includegraphics[width=\textwidth]{img/technologierecherche/Aufnahme/Laengsweg_Griff.jpg}
        \caption{Klemme über Längsweg des Hindernisses}
        \label{img:tech_Laengsweg_Griff}
    \end{minipage}
\end{figure}
\newpage
\begin{figure}[h!]
    \centering
    \begin{minipage}{0.45\textwidth}
        \centering
        \includegraphics[width=\textwidth]{img/technologierecherche/Aufnahme/Breiterweg_Griff.jpg}
        \caption{Klemme über Breitenweg des Hindernisses, kann modifiziert werden, um Berührungssensoren zu verwenden}
        \label{img:tech_Breiterweg_Griff}
    \end{minipage}
    \hfill
    \begin{minipage}{0.45\textwidth}
        \centering
        \includegraphics[width=\textwidth]{img/technologierecherche/Aufnahme/Vakuumgreifer.jpg}
        \caption{Aufnahme über Vakuumgreifer, Quelle: https://www.youtube.com/shorts/alxwWgzSVss} 
        \label{img:tech_Vakuumgreifer}
    \end{minipage}
\end{figure}

\begin{figure}[h!]
    \centering
    \begin{minipage}{0.45\textwidth}
        \centering
        \includegraphics[width=\textwidth]{img/technologierecherche/Aufnahme/Mantis.jpg}
        \caption{Über Längsweg, es wird ausgenutzt das es auf den Seiten Verbindungen zum Zusammenstecken der Hindernisse hat}
        \label{img:tech_Mantis}
    \end{minipage}
    \hfill
\end{figure}


\newpage
\subsubsection{Rotation / Transport}
\begin{figure}[h!]
    \centering
    \begin{minipage}{0.45\textwidth}
        \centering
        \includegraphics[width=\textwidth]{img/technologierecherche/Rotation/harponne.jpg}
        \caption{Eine Harpune artige Konstruktion, die die Löcher des Hindernisses ausnutzt}
        \label{img:tech_harponne}
    \end{minipage}
    \hfill
    \begin{minipage}{0.45\textwidth}
        \centering
        \includegraphics[width=\textwidth]{img/technologierecherche/Rotation/kran.jpg}
        \caption{Eine an einen Kran angelegte Konstruktion, Quelle: https://www.youtube.com/watch?v=VZRFHJfUkq4\&feature=youtu.be} 
        \label{img:tech_kran}
    \end{minipage}
\end{figure}

\begin{figure}[h!]
    \centering
    \begin{minipage}{0.45\textwidth}
        \centering
        \includegraphics[width=\textwidth]{img/technologierecherche/Rotation/seitlich_mit_räder.jpg}
        \caption{Für die Rotation wird das ganze Fahrzeug gewendet}
        \label{img:tech_seitlich_mit_räder}
    \end{minipage}
    \hfill
    \begin{minipage}{0.45\textwidth}
        \centering
        \includegraphics[width=\textwidth]{img/technologierecherche/Rotation/seitlich_mit_rotation.jpg}
        \caption{Ebenfalls an einen Kran angelegt, jedoch wird die Aufnahme von oben auf die Breite des Fahrzeuges durchgeführt, Quelle:https://www.youtube.com/watch?v=J7LGSNhFTU4} 
        \label{img:tech_seitlich_mit_rotation}
    \end{minipage}
\end{figure}
\newpage
\begin{figure}[h!]
    \centering
    \begin{minipage}{0.45\textwidth}
        \centering
        \includegraphics[width=\textwidth]{img/technologierecherche/Rotation/ueberkopf_griff_gelagert.jpg}
        \caption{Rotation am arm selber, damit das Hindernis wieder aufrecht steht.}
        \label{img:tech_ueberkopf_griff_gelagert}
    \end{minipage}
    \hfill
    \begin{minipage}{0.45\textwidth}
        \centering
        \includegraphics[width=\textwidth]{img/technologierecherche/Rotation/ueberkopf_objekt_gelagert.jpg}
        \caption{Lagerung am Hindernis } 
        \label{img:tech_ueberkopf_objekt_gelagert}
    \end{minipage}
\end{figure}

\begin{figure}[h!]
    \centering
    \begin{minipage}{0.45\textwidth}
        \centering
        \includegraphics[width=\textwidth]{img/technologierecherche/Rotation/waescheanlage.jpg}
        \caption{Beförderung des Objektes unter dem Fahrzeug}
        \label{img:tech_waescheanlage}
    \end{minipage}
    \hfill
\end{figure}





\newpage
\subsection{Elektrotechnik}
Im folgenden Abschnitt werden über die einzelnen elektrischen Varianten und deren Komponenten verglichen. Indem zu jeder Variante die Vor- und Nachteile aufgelistet werden. Dies vereinfacht später beim morphologischen Kasten, die möglichen Lösungswege.

\subsubsection{Antrieb}

In diesem Teil werden die verschiedenen elektrische Antriebe aufgelistet. 

\paragraph{DC-Motor}

Der DC-Motor ist ein Gleichstrom-Motor, der mit Bürsten am Kommutator arbeitet. 

\begin{minipage}[t]{0.48\textwidth}
\begin{items}
  \item [Vorteile]
  \item Einfache Drehzahlsteuerung
  \item Hohe Drehzahlen
  \item Hohes Drehmoment
\end{items}
\end{minipage}
\hfill
\begin{minipage}[t]{0.48\textwidth}
\begin{items}
  \item [Nachteile]
  \item Verschleiss der Bürsten
  \item EMV Störung durch Funkenbildung
  \item Schlechte Wärmeabführung
\end{items}
\end{minipage}

\paragraph{Brushless DC-Motor}

Brushless DC Motoren(BLDC) werden häufig im Modellbau oder Drohnen eingesetzt. Diese haben gegnüber normalen DC-Motoren keine Bürsten. Sie besitzen drei Wicklungen auf dem Stator und haben Im Rotor einen permanent Magneten. Die Spulen werden nun so angesteuert, dass es ein wechselndes Magnetfeld gibt. Dieses Drehfeld bringt den Rotor zur Dreh

\begin{minipage}[t]{0.48\textwidth}
\begin{items}
  \item [Vorteile]
  \item Einfache Drehzahlsteuerung
  \item Hohe Drehzahlen
  \item Hohes Drehmoment
\end{items}
\end{minipage}
\hfill
\begin{minipage}[t]{0.48\textwidth}
\begin{items}
  \item [Nachteile]
  \item Verschleiss der Bürsten
  \item EMV Störung durch Funkenbildung
  \item Schlechte Wärmeabführung
\end{items}
\end{minipage}






\subsubsection{Antrieb}
Als Antrieb würde sich hier ein einfacher DC Motor eignen. Dieser hat den Vorteil einfach in der Ansteuerung zu sein. Jedoch müsste man für die Auswertung des gefahrenen Weges, eventuell benötigt für die Hindernissbewältigung, ein Encoder anbringen und diesen auswerten. Der Schrittmotor kann dagegen präzise angesteuert werden und weiss daher immer seine Winkelposition. Ein Schrittmotor benötigt ein Treiber, ist jedoch in der Ansteuerung mit Positionserkennung einfacher. Der Brushless Motor ist wie ein kleiner Asynchronmotor. Dieser benötigt auch ein Treiber, jedoch muss dieser für die Positionserkennung auch ein Geber haben.

\subsubsection{Orientierung}
Bei der Orientierung geht es vor allem um das erkennen eines blockierten Weges. Der Infrarot Sensor sendet ein Optisches Signal und detektiert wie stark es reflektiert wurde. Damit kann er ein Objekt vor sich erkennen. So könnte man zwei Sensoren auf unterschiedlichen Höhen montieren und so zwischen Pylonen und Hindernissen unterscheiden. Infrarot funktioniert bei Partikel in der Luft ungenauer, da das in diesem Projekt nicht vorkommt, ist dies eine Möglichkeit. Auf dem gleichen Prinzip könnte man für die Erkennung auch ein Ultraschallsensor einsetzen. Der sendet ein Schallwellensignal aus und wertet das Echo aus. In windigen Umgebungen kann es zu Messfehlern führen, auch dies trifft in der Projektumgebung nicht auf. Um das Hindernis präzise zurückzustellen, wenn das Fahrzeug ein Weg zurücklegt, benötigt es ein Encoder. Ein absoluter Encoder kann gegenüber eines inkrementellen auch die Winkelposition des Motors zurückgeben. Demgegenüber liefert der inkrementelle für jede Umdrehung eine bestimmte Zahl an Impulsen. Für unsere Anwendung würde ein Inkrementeller genügen. 

\subsubsection{Sicherheit}
Als Sicherheit wird auf dem Fahrzeug ein gut zugänglicher Not-Aus Pilz Taster angebracht. Dieser muss das Fahrzeug stoppen und in einen sicheren Zustand bringen. Ausserdem muss Software technisch mögliche undefinierte Zustände abgefangen werden. Zum Beispiel wenn das Fahrzeug die Linie verlässt, wie es reagiert. Schaltet es den Antrieb aus oder irrt es ohne Orientierung weiter.

\subsubsection{Energiequelle}
Bei der Energiequelle bietet sich ein Akkumulator am besten an. Eine Technologie davon ist der Lithium-Polymer (LiPo) Akku. Diese haben eine hohe Leistungsdichte und können auch hohe Ströme liefern. Dafür sind sie empfindliche gegen Überladung, Tiefenentladung und Schlägen. Lithium-Ionen (Li-Ion) Akkus sind robuster im Bezug auf den Ladevorgang und haben auch die höhere Energiedichte. Dafür liefern sie die kleineren Ströme, haben dafür eine länger Lebensdauer. Nickel-Metallhydrid (NiMh) Akkus sind gegenüber der Lithium Technologien Umweltfreundlicher und sicherer im Umgang. Dafür ist ihre Energiedichte kleiner wie auch ihre maximale Stromabgabe. Eine autarke Lösung, wäre mittels Solarpanel. Damit ein solches System funktioniert, benötigt es trotz allem immer noch ein Akku um die Energie zwischen zu speichern. Daher ist das Solarpanel, besser als ein Feature zu verstehen.

